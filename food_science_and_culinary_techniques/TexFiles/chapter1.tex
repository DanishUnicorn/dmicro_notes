\chapter{Course Description}
\section{Content}
The aim of the course is to use a basic understanding of food chemistry and physics to obtain a scientific approach to cooking when various culinary techniques are applied during processing of foods.

The course includes a series of lectures giving a scientific description of foods as a chemical and physical system. It relates to proteins, lipids, carbohydrates as well as topics within general chemistry (inorganic and organic), acids and bases, and interactions of these components. The course provides an understanding of the culinary techniques used in the production of foods and highlights the effects of food processing on the chemical reactions leading to changes of flavour/taste and colour as well as the physical properties of the food in relation to changes in structure and functionality.

Practical kitchen exercises in preparation of foods will be used as a learning tool to the understanding of culinary techniques. The use of ingredients in various recipes will be evaluated and thereby demonstrate important experimental aspects of food processing and preparation. This will include an introduction to experimental design where recipes and preparations are varied, and the methods of evaluation are identified. The following afterlab discussions will reflect on the outcome of the experiments and correlate it to the scientific principles of the exercise.

The lectures and theoretical exercises will demonstrate how food components contributes to the functional properties in dry systems, crystalline states, emulsions, foams and other real food systems. During the practical kitchen exercises students will evaluate the different preparations in relation to texture, flavour/taste and colour, and explain the outcome according to the theory.

\section{Learning Outcome}
A student who has fulfilled the aim of the course should be able to:

\subsection{Knowledge}
\begin{highlight}
    \begin{itemize}
        \item Describe important chemical reactions and physical changes during processing of foods
        \item Describe carbohydrates, lipids and proteins and their basic functions and characteristics in food and point out the effects of culinary processes on physical, chemical and sensory conditions of food components
        \item Describe the effect of physical processes on the structure of food during cooking
        \item Identify factors of relevance for detection, perception and loss of aroma and flavour compounds in different solvents.
    \end{itemize}
\end{highlight}

\subsection{Skills}
\begin{highlight}
    \begin{itemize}
        \item Work in a gastronomic laboratory with specific culinary techniques and follow instructions to obtain a well-defined product
        \item Explain the changes in foods taking place during preparation of food from a chemical and a physical point of view
        \item Predict the outcome of various preparation methods and recipes based on a simple experimental design
        \item Evaluate a complex food and communicate the compositional structure of the product
        \item Evaluate the effect of various culinary techniques on the food structure and flavour
        \item Ability to perform simple statistical analyses using Excel or other related software
        \item Ability to give and receive peer feedback
        \item Ability to communicate scientific topics within food science and culinary techniques in (academic) English.
\end{itemize}
\end{highlight} 

\subsection{Competences}  
\begin{highlight}
    \begin{itemize}
        \item Plan experiments related to the effect of a culinary technique on the sensory properties of food
        \item Integrate scientific disciplines (food chemistry and food physics) in planning and evaluation of practical experiments
        \item Cooperate with other students on planning and performing practical exercises including written and oral evaluation of the theoretical outcome through afterlab discussions at plenary sessions.
    \end{itemize}
\end{highlight}

\section{Litterature}
See Absalon for a list of course literature.

\section{Recommended Academic Qualifications}
A basic knowledge of chemical reactions involving carbohydrates, proteins and lipids as well and basic statistics is highly recommended.

Academic qualifications equivalent to a BSc degree is recommended.

\section{Teaching and Learning Methods}
The teaching use a general understanding of food chemistry and physics in combination with practical kitchen exercises in a gastronomical laboratory to examine the influence of various processing methods on the food components. The practical kitchen exercises set the frame for group work and will be evaluated by afterlab discussions, problem-based learning and answering questions from the lecturers. The course also includes mandatory written assignments with peer feedback (student-student) based on the practical exercises. Specific practical exercises might be organised as take-home exercises where ingredients and tools will be provided.

\section{Remarks}
It is recommended to follow the course on the first year of the MSc Programme in Food Innovation and Health.

\section{Workload}
\begin{table}
    \centering
    \caption{A table with an overview over the workload for the course.}
    \label{tab:workload}
    \rowcolors{2}{white}{gray!7}
    \begin{tabular}{ l | c}
        \textbf{Category} & \textbf{Hours} \\ 
        \hline
        Lectures & 43 \\ 
        Preparation & 110 \\
        Theory exercises & 28 \\ 
        Practical exercises & 21 \\ 
        Exam & 4 \\ 
        \hline
        Total & 206 \\ 
    \end{tabular}
\end{table}

\section{Feedback Form}
\begin{highlight}
    \begin{itemize}
        \item Oral
        \item Collective
        \item Continuous feedback during the course of the semester
        \item Peer feedback (Students give each other feedback)
    \end{itemize}
\end{highlight}

\section{Sign Up}
Self Service at KUnet

http://www.science.ku.dk/english/courses-and-programmes/

https://www.science.ku.dk/english/continuing-and-professional-education/single-subject-courses/practical/

\section{Exam}
\newpage
\begin{table}[t]
    \centering
    \caption{The table shows the details of the course exam, as defined from the website of the University of Copenhagen.}
    \label{tab:course_details}
    \rowcolors{2}{white}{gray!7}
    \begin{tabular}{ l | >{\raggedright\arraybackslash}p{\textwidth - 5.8cm} }
        \textbf{Category} & \textbf{Details} \\ 
        \hline
        Credit & 7.5 ECTS \\ 
        Type of assessment & On-site written exam, 4 hours under invigilation \\ 
        Type of assessment details & Individual written 4 hour exam on specific topics based on the course curriculum. The on-site written exam is an ITX exam. See important information about ITX-exams at Study Information, menu point: Exams -> Exam types and rules -> Written on-site exams (ITX) \\ 
        Exam registration requirements & Approval of all assignments for all practical kitchen exercises. \\ 
        Aid & All aids allowed. The University will make computers available to students at the ITX-exam. Students are not permitted to bring digital aids like computers, tablets, mobile phones etc. Students are, however, allowed to bring a calculator. Books, notes, and similar materials can be brought in paper form or uploaded before the exam and accessed digitally from the ITX computer. Read more about this at Study Information. \\ 
        Marking scale & 7-point grading scale \\ 
        Censorship form & No external censorship. Several internal examiners. \\ 
        Re-exam & Same as ordinary exam. Possibility to edit and re-submit all non-approved assignments from the practical kitchen exercises two weeks before the re-exam. If 10 or fewer register for the re-examination the examination form will be oral. The oral exam will be 20 minutes in the course curriculum, no preparation time and all aids allowed. \\ 
    \end{tabular}
\end{table}

\textbf{Criteria for exam assessment}
See Learning Outcome.
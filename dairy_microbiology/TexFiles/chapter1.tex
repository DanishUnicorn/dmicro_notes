\chapter{Course Description}
\section{Content}
The course will contain lectures on:
\begin{highlight}
    \begin{itemize}
        \item Raw milk microflora
        \item Pathogenic and spoilage microflora
        \item Bio-protective cultures
        \item Starter cultures: primary (lactic acid bacteria) and adjunct (yeasts, moulds, propionibacterium, red smear microflora, lactic acid bacteria ripening cultures) starter cultures microflora as well as adventious microbflora (non-starter lactic acid bacteria) and their role in dairy products
        \item Bacteriophage and bacteriophage resistance in dairy environment
        \item Role of starter culture physiology on growth and end products (aroma formation, proteolysis, glycolysis, amino acid catabolism)
        \item Guest lectures from speakers representing major starter culture producers
    \end{itemize}
\end{highlight}

Laboratory and theoretical exercises will include:
\begin{highlight}
    \begin{itemize}
        \item Characterization of starter cultures using culture and culture independent techniques
        \item Detection and propagation of bacteriophages from dairy products
        \item Influence of starter culture propagation conditions on starter composition and activity
    \end{itemize}
\end{highlight}

\section{Learning Outcome}
The aim of the course is to give the students a thorough knowledge of roles and importance of microbes in manufacture of dairy products. Focus will be on primary and adjunct starter cultures for dairy products, and including the adventious microflora associated with cheese.

Through lectures and laboratory exercise the course will also give the students additional knowledge on microbiological analysis on organisms relevant for dairy products. Knowledge of the most important spoilage and pathogenic microbes in dairy products will also be obtained.
After completing the course the student should be able to:

\subsection{Knowledge}
\begin{highlight}
    \begin{itemize}
        \item Describe the composition of starter cultures and their use in dairy products

        \item Describe the secondary, adjunct and adventitious microflora found in cheese and their biochemical role in cheese ripening

        \item Comprehend the role of physiology of starter, secondary starter, adjunct cultures and non-starter bacteria in fermented dairy products

        \item Comprehend the principles for starter production
        
        \item Comprehend how bacteriophages influence fermented dairy product quality and how to detect them
        
        \item Display knowledge on culture dependent and culture independent techniques for quantification of microorganisms from dairy products
    \end{itemize}
\end{highlight}

\subsection{Skills}
\begin{highlight}
    \begin{itemize}
        \item Aplpy principles for species and strain identification of dairy organisms
        
        \item Analyse dairy product for relevant microorganisms using classical and molecular biology based methods and evaluate the putative source of them
        
        \item Compare literature information with own obtained data
\end{itemize}
\end{highlight} 

\subsection{Competences}  
\begin{highlight}
    \begin{itemize}
        \item Find, use and evaluate dairy microbiology literature in relation to dairy fermentation processes

        \item Cooperate with fellow students about literature and practical laboratory work

        \item Communicate and evaluate own data in relation to literature in writing and orally
    \end{itemize}
\end{highlight}

\section{Litterature}
See Absalon for a list of course literature. It will include textbooks, reviews and original litterature, presentations, notes and laboratory manuals.

\section{Recommended Academic Qualifications}
Qualifications within the field of microbiology of fermented food and beverages are recommended.

Academic qualifications equivalent to a BSc degree is recommended.

\section{Teaching and Learning Methods}
Lectures, theoretical and practical exercises, group work.

\section{Workload}
\begin{table}
    \centering
    \caption{A table with an overview over the workload for the course.}
    \label{tab:workload}
    \rowcolors{2}{white}{gray!7}
    \begin{tabular}{ l | c}
        \textbf{Category} & \textbf{Hours} \\ 
        \hline
        Lectures & 28 \\ 
        Preparation & 80 \\
        Theory exercises & 5 \\ 
        Practical exercises & 50 \\ 
        Study Groups & 20 \\
        Guidance & 10 \\
        Exam Preparation & 12 \\
        Exam & 1 \\ 
        \hline
        Total & 206 \\ 
    \end{tabular}
\end{table}

\section{Feedback Form}
\begin{highlight}
    \begin{itemize}
        \item Oral
        \item Collective
        \item Continuous feedback during the course of the semester
        \item Peer feedback (Students give each other feedback)
    \end{itemize}
\end{highlight}

\section{Sign Up}
Self Service at KUnet

http://www.science.ku.dk/english/courses-and-programmes/

https://www.science.ku.dk/english/continuing-and-professional-education/single-subject-courses/practical/

\section{Exam}
\begin{table}[h]
    \centering
    \caption{The table shows the details of the course exam, as defined from the website of the University of Copenhagen.}
    \label{tab:course_details}
    \rowcolors{2}{white}{gray!7}
    \begin{tabular}{ l | >{\raggedright\arraybackslash}p{\textwidth - 5.8cm} }
        \textbf{Category} & \textbf{Details} \\ 
        \hline
        Credit & 7.5 ECTS \\ 

        Type of assessment & Oral examination, 20 min \\ 

        Type of assessment details & Individual oral examination without time for preparation. At minimum 2 weeks before the examination, all examination questions (covering the essential issues of the course) are handed out. \\ 

        Aid & All aids allowed. \\ 

        Marking scale & 7-point grading scale \\ 

        Censorship form & No external censorship. Several internal examiners. \\ 

        Re-exam & Same as ordinary exam. \\ 
    \end{tabular}
\end{table}

\textbf{Criteria for exam assessment}
See Learning Outcome.

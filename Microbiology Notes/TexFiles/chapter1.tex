\chapter{Course description}

\section{Course content}

The course has major focus on the microorganisms involved in the processing of various fermented foods and beverages. The course includes the taxonomy of important microorganisms especially lactic acid bacteria (LAB) and yeasts covering both phenotypic characteristics and molecular typing techniques for their identification. Fermentation is introduced as a sustainable green technology and innovative technologies to improve fermented foods and beverages are considered.

Methodologies for isolation will be covered, including both culture dependent and culture independent techniques. Techniques such as high-throughput sequencing and digital programmes for bioinformatics are applied.

Various fermentation techniques is introduced covering the use of starter cultures and other fermentation techniques such as back-slopping. Further, the role of fermentation in sustainable food production and in prevention of food waste will be discussed.

An introduction to various fermented foods and beverages will be given including products such as cheese, bread, wine and beer as well as a number of traditional indigenous fermented food and beverages. Focus will additionally be on microbial interactions including topics such as quorum sensing, bacteriocin formation, etc.

\subsection{Learning objectives}

The objective of the course is to give the students a thorough knowledge on the microbiology behind production of fermented food and beverages and to give the students skills within isolation and identification of microorganisms occurring in these products. Additionally the students will be able to evaluate the functionalities and applications of microbial starter cultures.

\subsubsection{Knowledge}

\begin{highlight}
    \begin{itemize}
    \item Show overview of fermented food and beverages in general and the microorganisms involved in their production
    \item Describe important groups of microorganisms identified from fermented food and beverages
    \item Comprehend microbial taxonomic systems
    \item Describe microbial interactions and their importance in food systems
    \item Reflect on microbial cytology and physiology
    \item Define molecular techniques for identification and typing to species and strain level
\end{itemize}
\end{highlight}


 

\subsubsection{Skills} 

\begin{highlight}
    \begin{itemize}
        \item Apply procedures for isolation and identification of the predominant microorganisms in fermented food and beverages
        \item Explain at the molecular level the behaviour and interactions between various groups of microorganisms
        \item Assess the most important parameters leading to optimal product quality and food safety
        \item Apply food fermentation to develop innovative food products
    \end{itemize}
\end{highlight}
 

\subsubsection{Competences}  

\begin{highlight}
    \begin{itemize}
        \item Predict the composition of the microbiota of specific fermented food and beverages
        \item Discuss presumed functionalities of microorganisms in fermented food or beverages related to product quality and food safety
        \item Communicate and work independently on own data and discuss the results in relation to existing literature
    \end{itemize}
\end{highlight}


\subsection{Teaching and learning methods}

Lectures, theoretical and laboratory practicals. The lectures will introduce issues of importance for the understanding of microbial behaviour during production of fermented food and beverages. The theoretical and laboratory practicals will give the students practice on how to identify various microorganisms from fermented food and beverages including skills within various molecular techniques and digital tools for bioinformatic purposes. Knowledge on food innovation will be obtained throughout the course.

\subsection{Exam}

At the time of the oral examination, 20 min, one theoretical question is drawn, and the examination proceeds without preparation time. Two weeks before the exam, the questions will be given to the students. At the oral exam, the drawn question and the curriculum will account for 75\% of the grade. The discussion of the laboratory work and the experimental results will account for 25\% of the grade.

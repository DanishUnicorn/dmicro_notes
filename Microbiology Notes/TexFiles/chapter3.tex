\chapter{Laboratory Exercises}
\setlength{\headheight}{12.71342pt}
\addtolength{\topmargin}{-0.71342pt}

\section*{Introduction}
In this chapter a summary of the laboratory exercises is given. The exercises are designed to give the student a practical understanding of the theoretical concepts discussed in the previous chapters. The exercises are divided into two parts: the first part deals with the basic techniques of microbiology, while the second part deals with the identification of bacteria. 

\section{Day 1}
\subsection*{Sampling and Dilution}
The exercise began with aseptic sampling of Kefir. 10.0 g of Kefir was transferred to a sterile stomacher bag, followed by the addition of 90.0 ml of sterile 2\% $Na_3$-citrate solution. The mixture was homogenized in a stomacher for 1 minute at high speed, resulting in a $10^{-1}$ dilution.
\vspace{0.5em}
A dilution series was prepared by transferring 1.0 ml of the $10^{-1}$ dilution into 9.0 ml sterile Saline Peptone Water (SPO) to obtain a $10^{-2}$ dilution. This process was repeated until a $10^{-6}$ dilution was achieved.

\subsection*{Plating}
For enumeration of Lactic Acid Bacteria (LAB), 100 $\mu$l from each dilution ($10^{-1}$ to $10^{-6}$) was transferred to MRS agar plates. The inoculum was spread using a sterile spatula. Plates were then transferred to anaerobic jars with gas-generating kits for anaerobic conditions.

\subsection*{Incubation}
The inoculated plates were incubated anaerobically, using anaerobic gas generator bags in containers, at 30\textdegree C for 3-5 days.

\section{Day 2}
\subsection*{CFU Determination}
After the incubation period, colonies on plates containing between 20-200 colonies were counted. The Colony Forming Units (CFU) per gram of Kefir were calculated using the weighted mean Equation \ref*{eq:WeightMean}:

\begin{equation}
    \frac{CFU}{g} =  \frac{\Sigma c}{(n_1 + 0.1 n_2)} \cdot \frac{1}{V} \cdot \frac{1}{d_1}
    \label{eq:WeightMean}
\end{equation}
\vspace{0.5em}
Where $\Sigma_c$ is the sum of colonies counted on all plates, $n_1$ is the number of plates in lower dilution, $n_2$ is the number of plates in higher dilution, and d is the dilution factor of the lower dilution.

\subsection*{Isolation of Colonies}
Eight colonies were randomly selected from suitable plates for further analysis. Each colony was transferred to 100 $\mu$L of sterile SPO in an Eppendorf tube and vortexed.

\subsection*{Purification of Isolates}
For each isolate, 10 $\mu$L of the bacterial suspension was streaked onto half of an MRS agar plate to obtain single colonies. Additionally, 25 $\mu$L of the suspension was transferred to a labeled test tube containing MRS broth.
\vspace{0.5em}
The streaked plates were incubated anaerobically at 30\textdegree C, while the broth cultures were incubated aerobically at 30\textdegree C for 1-2 days.

\section{Day 3}
\subsection*{Macro- and Micro-morphological Examinations}
Purified isolates were examined for their macro-morphological characteristics on MRS agar, including colony size, surface texture, margin shape, elevation, and pigmentation.
\vspace{0.5em}
Micro-morphological examinations were conducted using light microscopy to observe cell shape, arrangement, and motility.

\subsection*{Preliminary Phenotypic Tests}
The following tests were performed on each isolate:

\begin{itemize}
    \item Gram staining using the KOH method
    \item Catalase test using 3\% $H_2O_2$
    \item Oxidase test using oxidase disks
    \item $CO_2$ production test using MRS broth with inverted Durham tubes
\end{itemize}

\section{Day 4}
\subsection*{DNA Extraction}
DNA was extracted from five selected isolates using the Instagene kit (Bio-Rad) following the manufacturer's protocol with slight modifications. The extracted DNA was stored at -20\textdegree C.

\subsection*{Rep-PCR/(GTG)5 Fingerprinting}
Rep-PCR was performed using the (GTG)5 primer. The PCR mixture was prepared, and the thermo-cycling program was set as follows: initial denaturation at 95\textdegree C for 5 min, followed by 30 cycles of 95\textdegree C for 30 sec, 45\textdegree C for 60 sec, and 65\textdegree C for 8 min, with a final extension at 65\textdegree C for 16 min.

\subsection*{Gel Electrophoresis}
PCR products were separated on a 1.5\% agarose gel in 0.5 x TBE buffer. Electrophoresis was run at 120 V for 5 hours.

\section{Day 5}
\subsection*{Rep-PCR Profile Analysis}
Rep-PCR profiles were visually analyzed to group isolates with similar patterns.

\subsection*{16S rRNA Gene Amplification}
Based on the Rep-PCR groupings, representative isolates were selected for 16S rRNA gene sequencing. PCR amplification of the 16S rRNA gene was performed using primers 27F and 1540R. The PCR conditions were: initial denaturation at 95\textdegree C for 5 min, followed by 35 cycles of 95\textdegree C for 30 s, 60\textdegree C for 30 s, and 72\textdegree C for 120 s, with a final extension at 72\textdegree C for 10 min.

\subsection*{PCR Product Verification}
The amplified 16S rRNA gene products were verified by running 5 $\mu$L on a 1.5\% agarose gel at 120 V for 30 minutes.

\section{Day 6}
\subsection*{Sequence Analysis}
The obtained 16S rRNA gene sequences were analyzed using the BLAST tool on the NCBI website and the EzTaxon database for identification of the isolates to the species level.

\subsection*{Literature Review}
Literature reviews was conducted to identify additional phenotypic tests that could confirm the sequencing results or further differentiate the isolates to the strain level.

\section{Results}
\subsection{Colony Forming Units (CFU)}
The spread plate method yielded countable colonies on the MRS agar plates. $d_1$ was of $10^{-3}$ from the dilution series and was determined to have a yield of $c_1$ of 180 colonies. $d_2$ was of 10$^{-4}$ and had a yield of $c_2$ of 29 colonies. The CFU per gram of Kefir was calculated using Equation \ref*{eq:WeightMean} as 1.9 x $10^6$.

\subsection{Rep-PCR Fingerprinting}
The Rep-PCR analysis using the (GTG)5 primer revealed distinct banding patterns for the isolated strains. Notable observations include:

Isolates 2 and 7 exhibited similar fingerprint patterns, suggesting they may belong to the same species.
Isolates 4 and 8 also showed similar patterns, indicating another potentially shared species.
The blank control showed no bands, confirming the absence of contamination or non-specific amplification.

\subsection{16S rRNA Sequencing Results}
\subsubsection{Isolate No. 8}
Sequencing of the 16S rRNA gene for Isolate No. 8 revealed high similarity to several \textit{Lactococcus} species:

\begin{itemize}
    \item \textit{Lactococcus lactis} subsp. \textit{lactis} (99.80\% similarity)
    \item \textit{Lactococcus lactis} subsp. \textit{hordniae} (99.66\% similarity)
    \item \textit{Lactococcus cremoris} subsp. \textit{tructae} (99.18\% similarity)
    \item \textit{Lactococcus cremoris} subsp. \textit{cremoris} (99.11\% similarity)
\end{itemize}

\vspace{0.5em}
Phenotypic characteristics of Isolate No. 8 included:

\begin{itemize}
    \item Gram-positive
    \item Homo-fermentative
    \item Catalase-negative
    \item Oxidase-negative
    \item Cocci arranged in pairs or short chains
    \item Non-motile
\end{itemize}

\subsubsection{Isolates No. 1 \& 7}
16S rRNA sequencing for Isolates No. 1 and 7 showed high similarity to several \textit{Leuconostoc} species:

\begin{itemize}
    \item \textit{L. mesenteroides} subsp. \textit{mesenteroides} (98.95\% similarity)
    \item \textit{L. mesenteroides} subsp. \textit{dextranicum} (98.95\% similarity)
    \item \textit{L. mesenteroides} subsp. \textit{cremoris} (98.86\% similarity)
    \item \textit{L. pseudomesenteroides} (98.77\% similarity)
    \item \textit{L. falkenbergense} (98.67\% similarity)
    \item \textit{L. suionicum} (98.86\% similarity)
\end{itemize}

\vspace{0.5em}
Phenotypic characteristics of Isolates No. 1 \& 7 included:

\begin{itemize}
    \item Gram-positive
    \item Hetero-fermentative
    \item Catalase-negative
    \item Oxidase-negative
    \item Cocci arranged in pairs or short chains
    \item Non-motile
\end{itemize}

\subsection{Comparison with Literature Findings}
The identified species in our kefir sample partially aligned with those commonly reported in the literature. While our analysis identified various \textit{Lactococcus} and \textit{Leuconostoc} species, the literature also reports the presence of:
\begin{itemize}
    \item \textit{Streptococcus thermophilus}
    \item \textit{Lactobacillus acidophilus}
    \item \textit{Lactobacillus casei}
    \item \textit{Lactobacillus kefiranofaciens}
    \item \textit{Lactobacillus kefiri}
\end{itemize}

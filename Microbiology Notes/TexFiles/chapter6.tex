\chapter{Exam}

\section{Regarding the Curriculum}
The questions asked will cover the course curriculum in general. Be aware that we specifically expect you to be able to describe the principles behind molecular methodologies for identification (species) and typing (subspecies/strain) of both pro- and eukaryotes.  

\section{Quideline for exam questions}
You need to prepare for all of the 13 exam questions before the exam. You are allowed to use notes when presenting the exam question and to bring printed versions of your lab group slides. It is up to you to organize the oral presentation (no power point or shared screen are allowed). We recommend that you prepare keywords, and practice what you want to say based on the keywords. Emphasize the points you find most relevant and would like to discuss. Please notice, that it is not allowed to read aloud from a written text.  

The grade will primarily reflect the your ability to analyze and discuss the exam question and other topics related to the curriculum. 

\section{Exam Questions}

\subsection{AAB}
\subsubsection*{Question}
Acetic acid bacteria (AAB) are important in several food and biotechnological processes. Define AAB and name the most important AAB relevant to food. Describe at least two food processes where AAB play an essential role and explain their functions in these products/processes.

\subsection{AAB}
\subsubsection*{Question}
Name at least one suitable substrate for isolation of AAB and explain the selective and/or indicative principle(s) behind the substrate. Name at least three phenotypic features of AAB that can be used to differentiate AAB from other microorganisms (LAB, yeasts) occasionally growing on AAB-selective substrates. Give an overview of methods that can be used to identify the different isolates to species level. 

\subsection{Yeast}
\subsubsection*{Question}
Give 2-3 examples on foods and/or beverages that are fermented by yeasts. Explain whether single or multiple species participate in the fermentation and whether the yeast(s) are added as starter culture(s) or occur spontaneously. Describe the functions of the yeasts in the products. Explain how yeast can be identified to species level and give examples of typing techniques for strain characterization. 

\subsection{Yeast}
\subsubsection*{Question}
Describe the micro- and macromorphological characteristics of the species Saccharomyces cerevisiae and explain how this species differs genetically from Saccharomyces pastorianus. Give examples on food and/or beverages that favor the growth of Saccharomyces spp. and mention potential yeast interactions with other types of microorganisms. 

\subsection{Bacillus}
\subsubsection*{Question}
Identify the most commonly occurring species of aerobic, endospore-forming microorganisms found in alkaline fermented foods. Mention at least 2-3 examples of desired technological properties of Bacillus spp. and how these technological properties influence the quality of the final product. Discuss shortly the implications of spontaneous as compared to controlled fermentations for the quality and safety of the final product.  

\subsection{LAB}
\subsubsection*{Question}
The former Lactobacillus genus has been split into 25 new genera. Among these new genera are \textit{Companilactobacillus}, \textit{Lactiplantibacillus}, \textit{Lactobacillus}, \textit{Lacticaseibacillus}, \textit{Levilactobacillus}, \textit{Lentilactobacillus}, and \textit{Limosilactobacillus}. From these genera, mention two species of relevance for fermentation of food or beverages, one of them being a specialist and the other a generalist; describe their habitat, typical phenotypes, and their contribution to the fermentation. Describe how you would identify the species by use of phenotypic and molecular methodologies.

\subsection{LAB}
\subsubsection*{Question}
Explain the strain definition for bacteria. Describe the methods you will use for strain typing and the principles behind these methods (select 2-3 methods). Additionally, explain how these methods can be used to determine whether different strains of \textit{Lacticaseibacillus casei} used as ripening cultures during cheese production are able to grow in the cheese.

\subsection{LAB}
\subsubsection*{Question}
Give an overview of the antimicrobial compounds produced by LAB. How do you detect the inhibitory compounds and how do you investigate which types of inhibitory compounds are produced? How will you determine that an inhibitory compound is a bacteriocin? Classify bacteriocins and give examples of their inhibition spectrum and mode of action. 

\subsection{Moulds}
\subsubsection*{Question}
Mention 2-3 mould genera and methodologies for their identification to species level. Give examples (2-3) of how moulds are used in fermentation of foods and beverages. Discuss the positive and negative characteristics of moulds in food production. 

\subsection{Microbial Interactions/Quorum Sensing}
\subsubsection*{Question}
Describe the principles behind quorum sensing (QS) and give examples of microbial properties (traits) that are under QS control. Explain how QS differs from other types of microbial interactions. Mention 2-3 types of microbial interactions which are not under QS control and explain their mode of action.  

\subsection{Probiotics}
\subsubsection*{Question}
Give a definition of probiotics and name 2-3 species where strains are commonly applied as probiotics. Give examples on properties that a potential probiotic candidate can fulfil. Line out diverse ways of investigating these probiotic properties. Probiotics might have beneficial effects against diarrhea. What are the proposed mechanisms behind this effect?

\subsection{Plant-based Fermented Foods/Health}
\subsubsection*{Question}
Describe how microorganisms, through fermentation of plant-based foods, can benefit nutritional quality and human health. Include at least 2-3 different effects and discuss their underlying mechanisms. 

\subsection{Sequencing Methods}
\subsubsection*{Question}
Define culture-independent and culture-dependent techniques for microbial characterization of fermented food and describe advantages and disadvantages of the two approaches. Give 1-2 examples on how you can apply Nanopore amplicon sequencing for species identification. Discuss the advantages and disadvantages of amplicon sequencing compared to whole genome sequencing. 


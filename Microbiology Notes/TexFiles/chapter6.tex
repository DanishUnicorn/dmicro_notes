\setlength{\headheight}{12.71342pt}
\addtolength{\topmargin}{-0.71342pt}

\chapter{Exam}

\section{Regarding the Curriculum}
The questions asked will cover the course curriculum in general. Be aware that we specifically expect you to be able to describe the principles behind molecular methodologies for identification (species) and typing (subspecies/strain) of both pro- and eukaryotes.  

\section{Guideline for exam questions}
You need to prepare for all of the 13 exam questions before the exam. You are allowed to use notes when presenting the exam question and to bring printed versions of your lab group slides. It is up to you to organize the oral presentation (no power point or shared screen are allowed). We recommend that you prepare keywords, and practice what you want to say based on the keywords. Emphasize the points you find most relevant and would like to discuss. Please notice, that it is not allowed to read aloud from a written text.  

The grade will primarily reflect the your ability to analyze and discuss the exam question and other topics related to the curriculum. 

\section{Exam Questions}

\subsection{AAB-1}
\subsubsection*{Question}
Acetic acid bacteria (AAB) are important in several food and biotechnological processes. Define AAB and name the most important AAB relevant to food. Describe at least two food processes where AAB play an essential role and explain their functions in these products/processes.

\subsubsection*{Keywords}
Table \ref{tab:KW-AAB1} shows the keywords for this question and reflects the answer written in the following subsection.
\begin{table}[h]
    \centering
    \caption{Keywords for AAB-1} 
    \label{tab:KW-AAB1}
    \begin{tabular}{l|l}
        \textbf{Topic} & \textbf{Keywords} \\
        \hline
        AAB Overview & Gram-negative, oxidize ethanol \\
        Characteristics & Catalase positive, oxidize alcohols, mesophilic \\
        Notable Genera & \textit{Acetobacter}, \textit{Gluconobacter} \\
        Classification & \textit{Proteobacteria}, \textit{Acetobacteraceae} \\
        Phenotypical Differentiation & Flagellation, growth on acetic acid \\
        Identification & PCR, 16S rRNA, MALDI-TOF and WG sequencing\\
        Cultivation & AE and MYA mediums, incubation conditions \\
        Cultivation Challenges & Growth competition \\
        Technological Properties & Vinegar production, alcohol oxidation \\
        Fermented Foods & Vinegar, coffee, cocoa, kombucha, sourdough \\
    \end{tabular}
\end{table}

\subsubsection*{Answer}
\subsubsubsection*{AAB}
AAB belong to the family \textit{Acetobacteraceae} and are primarily Gram-negative, with a few Gram-variable exceptions. They are obligate aerobes that require oxygen for growth and are capable of oxidizing ethanol into acetic acid, a key trait in food and biotechnological processes.

\subsubsubsection*{Characteristics}
\begin{itemize}
    \item Gram-negative, ellipsoid to rod-shaped.
    \item Catalase positive, oxidase negative.
    \item Mesophilic, with optimal growth between 25-30\textdegree C, though some species can tolerate higher temperatures.
    \item pH optimum is between 5-6.5, though some AAB can grow at pH as low as 3.
    \item Oxidize sugars and alcohols to organic acids, primarily ethanol to acetic acid (\textit{Asaia} spp. being the exception, as it does not oxidize ethanol).
\end{itemize}

\subsubsubsection*{Notable genera} 
\textit{Acetobacter} (found in sugar-rich environments) and \textit{Gluconobacter} (found in alcohol-rich environments).

\subsubsubsection*{Classification}
AAB are classified as follows:
\begin{itemize}
    \item Phylum: \textit{Proteobacteria}
    \item Class: \textit{Alpha Proteobacteria}
    \item Family: \textit{Acetobacteraceae}
    \item Genera: \textit{Acetobacter}
\end{itemize}
\subsubsubsection*{Phenotypical characteristics differentiations}
Here are some common characteristics that can be used to differentiate AAB from each other:
\begin{table}[h]
    \centering
    \caption{Differential characteristics of AAB} 
    \label{tab:EA-AAB1.1}
            \begin{tabularx}{\textwidth}{ll} % Adjust this value as needed
                \multicolumn{2}{c}{\textbf{Characteristics}} \\
                \hline
                Flagellation & Oxidation of ethanol to acetic acid \\
                Oxidation of acetic acid to CO\textsubscript{2} and H\textsubscript{2}O & Growth on 0.35\% acetic acid-containing medium \\
                 Growth on methanol & Growth on \textsubscript{D}-mannitol\\
                Growth in the presence of 30\% \textsubscript{D}-glucose & Production of cellulose \\
                Production of levan-like mucous substance from sucrose & Fixation of molecular nitrogen\\
                Ketogenesis (dihydroxyacetone) from glycerol & \\
                \hline
            \end{tabularx}

            \begin{tabular}{c}
                \textit{Acid production from:} \\
                \hline
                \textsubscript{D}-mannitol\\
                Glycerol\\
                Raffinose \\
                Cellular fatty acid type \\
                Ubiquinone type \\ 
                DNA base composition [mol\% G+C] \\
            \end{tabular}
\end{table}

\subsubsubsection*{Identification}
Identification of AAB can be done using phenotypic tests, which are tedious and difficult to perform. Therefore many prefer to use molecular biology-based tests instead. PCR-RFLP-analysis, Rep-PCR, 16S rRNA gene sequencing, MALDI-TOF, MLSA, and full genome sequencing are some of the methods used for identification.

\subsubsubsection*{Cultivation}
Cultivation of AAB is quite difficult, they often occur in environments where especially LAB and yeast also thrive starting a war of environment domination. There are various growth mediums that can be used to isolate AAB, such as  acetic acid-ethanol (AE) or malt extract-yeast extract-acetic acid(MYA).

The AAB should be incubated for 3-6 days at temperatures between 25-30\textdegree C. The colonies will appear as small, round, and convex with a smooth surface. They are usually white, cream, or yellow in color.

\subsubsubsection*{Technological properties}
AAB oxidises ethanol to acetic acid, which is a key trait in vinegar production. This exothermal reaction is carried out by the enzyme alcohol dehydrogenase, which is present in the periplasmic space of the AAB ($\text{ethanol} + \text{O}_2 \rightarrow \text{acetic acid} + \text{H}_2\text{O}$). The acetic acid produced can be further oxidized to $\text{CO}_2$ and $\text{H}_2\text{O}$ by some species, e.g. \textit{acetobacter} spp., \textit{acidomonas} spp., and \textit{gluconacetobacter} spp., which over-oxidises the acetic acid.
Futhermore, some strains of AAB can oxidise alcohols into sugars, e.g. mannitol into fructose. 

\textit{Gluconobacter} is industrially important and it produces \textsubscript{L}-sorbose from \textsubscript{D}-sorbitol, \textsubscript{D}-gluconic acid, 5-keto- and 2-keto-\textsubscript{D}-gluconate from \textsubscript{D}-glucose, and di-hydroxyacetone from glycerol.

\subsubsubsection*{Fermented foods}
Some examples of fermented foods where AAB play an essential role are:
\begin{itemize}
    \item Vinegar
    \subitem Generally \textit{Acetobacter pasteurianus}, but also other species
    \item Coffee
    \subitem Role unknown
    \item Cocoa
    \subitem Microbial succession
    \item Lambic
    \item Kombucha
    \subitem Cellulose layer
    \item Some sourdoughs
\end{itemize}

\subsection{AAB-2}
\subsubsection*{Question}
Name at least one suitable substrate for isolation of AAB and explain the selective and/or indicative principle(s) behind the substrate. Name at least three phenotypic features of AAB that can be used to differentiate AAB from other microorganisms (LAB, yeasts) occasionally growing on AAB-selective substrates. Give an overview of methods that can be used to identify the different isolates to species level. 

\subsubsection*{Keywords}
Table \ref{tab:KW-AAB2} shows the keywords for this question and reflects the answer written in the following subsection.
\begin{table}[h]
    \centering
    \caption{Keywords for AAB-2} 
    \label{tab:KW-AAB2}
    \begin{tabular}{l|l}
        \textbf{Topic} & \textbf{Keywords} \\
        \hline
        AAB Isolation Substrates & AE, MYA, GYC, MYGP \\
        Selective Substrate Principles & Ethanol, acetic acid, inhibition of LAB and yeasts \\
        AAB Phenotypic Differentiation & Gram-negative, ethanol oxidation, obligate aerobes \\
        Species Identification Methods & CR-RFLP, rep-PCR, 16S rRNA, MALDI-TOF, MLSA \\
    \end{tabular}
\end{table}

\subsubsection*{Answer}
\subsubsubsection*{Suitable substrate for isolation of AAB:}
"There are various growth mediums that can be used to isolate AAB, such as acetic acid-ethanol (AE) or malt extract-yeast extract-acetic acid (MYA)."
This is on page 138(main).
GYC - Glucose Yeast Calcium Carbonate Agar is also a suitable substrate for isolation of AAB. (Niclas)

A suitable substrate for yeast isolation mentioned is "MYGP (Malt Yeast Glucose Peptone) agar," which is used for isolating yeasts from products like apple cider. However, the document does not specify a substrate exclusively for Acetic Acid Bacteria (AAB). This information is found on page 21(LabProto).

A suitable substrate for the isolation of Acetic Acid Bacteria (AAB) is malt extract-yeast extract-acetic acid (MYA) medium, which selectively supports the growth of AAB due to their ability to grow in the presence of acetic acid while inhibiting other bacteria and yeasts (main) 

\subsubsubsection*{Selective and/or indicative principles behind the substrate}
"The acetic acid-ethanol medium (AE) is selective because acetic acid bacteria (AAB) can oxidize ethanol into acetic acid, a key trait of these organisms. The presence of ethanol and acetic acid in the medium inhibits the growth of many other microorganisms, including lactic acid bacteria (LAB) and yeasts, while promoting the growth of AAB."
This is also on page 138(main).

MYGP medium is supplemented with antibiotics (e.g., chloramphenicol, chlortetracycline) to prevent bacterial growth, thus allowing for selective isolation of yeasts. This medium may indirectly help in isolating AAB, as they can resist lower pH environments. This is detailed on page 23(LabProto).

\subsubsubsection*{Phenotypic features of AAB that can be used to differentiate AAB from other microorganisms}
"AAB are Gram-negative."
"They are obligate aerobes."
"They oxidize ethanol to acetic acid."
"Some strains can oxidize acetic acid further to $CO_2$ and $H_2O$."
"They grow on 0.35\% acetic acid-containing medium."
This is also on page 138(main).

While AAB are not directly mentioned, LAB are described as catalase-negative, which is one feature that helps differentiate LAB from other bacteria. Gram-positive bacteria are differentiated through a Gram test, and colony morphology descriptions are provided for yeast identification. This information is found on pages 30-32(LabProto).

\subsubsubsection*{Methods to identify different isolates to species level}
"AAB identification often relies on molecular biology techniques due to the tedious nature of phenotypic tests. Methods such as PCR-RFLP analysis, rep-PCR (often using (GTG)5), and sequencing of the 16S rRNA gene or its internal transcribed spacer are commonly used. Other techniques include MALDI-TOF mass spectrometry and multi-locus sequence analysis (MLSA), which provide more precise identification of AAB species."
This information can be found on page 22(main).

"Rep-PCR/(GTG)5 fingerprinting" is used for grouping and tentative identification, generating species-specific patterns that can be analyzed through software such as Bionumerics (page 10) (LabProto).

Sequencing of 16S rRNA (for bacteria) and 26S rRNA (for yeasts) is used for species-level identification. The PCR process and sequencing are described on pages 13-15 (LabProto).

Molecular techniques like BLAST searching are applied for further species identification based on sequence data, as outlined on page 16 (LabProto).

\subsection{Yeast-1}
\subsubsection*{Question}
Give 2-3 examples on foods and/or beverages that are fermented by yeasts. Explain whether single or multiple species participate in the fermentation and whether the yeast(s) are added as starter culture(s) or occur spontaneously. Describe the functions of the yeasts in the products. Explain how yeast can be identified to species level and give examples of typing techniques for strain characterization. 

\subsubsection*{Keywords}
Table \ref{tab:KW-Yeast1} shows the keywords for this question and reflects the answer written in the following subsection.

\subsubsection*{Answer}
\subsubsubsection*{Examples of foods and beverages fermented by yeasts}
\textbf{Wine}
\begin{itemize}
    \item Yeast Participation: Wine fermentation typically involves multiple species, with \textit{Saccharomyces cerevisiae} being the predominant yeast used as a starter culture. However, non-\textit{Saccharomyces} yeasts such as \textit{Hanseniaspora}, \textit{Pichia}, and \textit{Brettanomyces} can also contribute to fermentation, especially in spontaneous fermentations \cite*{L9-ISAPP}.
    \item Functions: Yeasts convert sugars from grapes into alcohol and carbon dioxide, while also producing various metabolites that contribute to the wine's flavour, aroma, and mouthfeel \cite*{L6-Yeasts,L9-ISAPP}.
\end{itemize}

\begin{table}[h]
    \centering
    \caption{Keywords for Yeast-1} 
    \label{tab:KW-Yeast1}
    \begin{tabular}{l|l}
        \textbf{Topic} & \textbf{Keywords} \\
        \hline
        Wine Fermentation & \textit{Saccharomyces}, non-\textit{Saccharomyces}, alcohol, aroma \\
        Beer Fermentation & \textit{S. cerevisiae}, \textit{S. pastorianus}, spontaneous, flavour \\
        Sourdough Fermentation & \textit{Saccharomyces exiguus}, \textit{Candida milleri}, lactic acid bacteria, leavening \\
        Yeast Identification Methods & Morphology, biochemical, molecular, phenotypic, genomic \\
    \end{tabular}
\end{table}

\vspace*{0.5em}
\textbf{Beer}
\begin{itemize}
    \item Yeast Participation: Similar to wine, beer is predominantly fermented by \textit{Saccharomyces pastorianus} or \textit{S. cerevisiae}, depending on the beer style (lager or ale, respectively). Multiple species may be involved in spontaneous fermentation, including \textit{Brettanomyces} and \textit{Lactobacillus} \cite*{L6-Yeasts, L9-ISAPP}.
    \item Functions: Yeasts in beer fermentation are responsible for converting fermentable sugars into alcohol and carbon dioxide, and they also produce compounds that enhance the beer's aroma and flavour profile \cite*{L6-Yeasts, L9-ISAPP}.
\end{itemize}

\vspace*{0.5em}
\textbf{Sourdough Bread}
\begin{itemize}
    \item Yeast Participation: Sourdough fermentation involves a complex community of microorganisms, primarily \textit{Saccharomyces exiguus} and \textit{Candida milleri}, alongside lactic acid bacteria. Yeasts may originate spontaneously from the environment or be introduced through starter cultures \cite*{L6-Yeasts}.
    \item Functions: In sourdough, yeasts ferment sugars to produce carbon dioxide, which leavens the bread, while lactic acid bacteria produce lactic acid, contributing to the sour flavour .
\end{itemize}

\subsubsubsection*{Identification of yeasts to species level}
Yeast species can be identified through various methods, including:
\begin{itemize}
    \item \textbf{Morphological Identification}: Traditional methods involve observing the size, shape, and reproduction modes (budding, fission) under a microscope \cite*{L6-Yeasts,L9-Coryn}.
    \item \textbf{Biochemical Tests}: Metabolic tests that assess the yeast's ability to ferment sugars or produce specific enzymes can help distinguish between species \cite*{L6-Yeasts,L9-Coryn}.
    \item \textbf{Molecular Techniques}: PCR amplification of specific genes (e.g., ITS region, 26S rRNA) followed by sequencing provides accurate identification. Additionally, molecular typing techniques such as Restriction Fragment Length Polymorphism (RFLP), Randomly Amplified Polymorphic DNA (RAPD), and Multilocus Sequence Typing (MLST) are used for strain characterization \cite*{L9-ISAPP,L6-Yeasts}.
    \item \textbf{Phenotypic Methods}: Techniques like the API 20C kit can determine yeast species based on fermentation and assimilation patterns of various substrates \cite*{L6-Yeasts}.
    \item \textbf{Genomic Methods}: Whole-genome sequencing can provide detailed information on the genetic makeup of yeasts, aiding in species identification and strain characterization \cite*{L9-ISAPP}.
\end{itemize}

\subsection{Yeast-2}
\subsubsection*{Question}
Describe the micro- and macromorphological characteristics of the species \textit{Saccharomyces cerevisiae} and explain how this species differs genetically from \textit{Saccharomyces pastorianus}. Give examples on food and/or beverages that favor the growth of \textit{Saccharomyces} spp. and mention potential yeast interactions with other types of microorganisms. 

\subsubsection*{Keywords}
Table \ref{tab:KW-Yeast2} shows the keywords for this question and reflects the answer written in the following subsection.
\begin{table}[h]
    \centering
    \caption{Keywords for Yeast-2} 
    \label{tab:KW-Yeast2}
    \begin{tabular}{l|l}
        \textbf{Topic} & \textbf{Keywords} \\
        \hline
        Micro-morphology of \textit{S. cerevisiae} & Globose, ovoid, budding \\
        Macro-morphology of \textit{S. cerevisiae} & Creamy colonies, smooth, 1-4 mm \\
        Genetic Differences & High ethanol tolerance, lager fermentation \\
        Growth in Wine & \textit{S. cerevisiae}, anaerobic, sugars $\leftarrow$ ethanol \\
        Growth in Beer & \textit{S. pastorianus} (lager), \textit{S. cerevisiae} (ale), malt sugars \\
        Growth in Sourdough & \textit{S. exiguus}, \textit{C. milleri}, carbohydrates, anaerobic \\
        Yeast-Bacteria Interactions & \textit{Lactobacillus}, sourdough, fermentation efficiency, flavour, texture \\
    \end{tabular}
\end{table}

\subsubsection*{Answer}
\subsubsubsection*{Micro- and macromorphological characteristics of \textit{Saccharomyces cerevisiae}}
\begin{itemize}
    \item \textbf{Micro-morphological characteristics}: \textit{Saccharomyces cerevisiae} cells are typically globose to ovoid, measuring approximately 5 to 10 $\mu$m in diameter. They reproduce asexually by budding, where daughter cells form as small protrusions on the surface of the parent cell \cite*{LabManual}.
    \item \textbf{Macro-morphological characteristics}: On solid media, \textit{S. cerevisiae} forms creamy to white colonies, which can be smooth and glistening or slightly dull, depending on growth conditions. The colony size can vary from pinpoint (<1 mm) to larger sizes (>4 mm) \cite*{LabManual}.
\end{itemize}

\subsubsubsection*{Genetic differences between \textit{S. cerevisiae} and \textit{S. pastorianus}}
\textit{Saccharomyces cerevisiae} and \textit{Saccharomyces pastorianus} differ primarily in their genetic makeup, particularly in their adaptability to fermentation conditions:
\begin{itemize}
    \item \textit{S. cerevisiae} is predominantly used in high-temperature fermentations (e.g., ales) and exhibits high ethanol tolerance \cite*{L9-Coryn,L6-Yeasts}.
    \item \textit{S. pastorianus}, a hybrid of \textit{S. cerevisiae} and \textit{S. eubayanus}, is adapted for lager fermentation at lower temperatures and has distinct genetic variations, particularly in genes associated with cold tolerance and lagging fermentation characteristics \cite*{L9-Coryn,L6-Yeasts}.
\end{itemize}

\subsubsubsection*{Foods and beverages favouring the growth of \textit{Saccharomyces} spp.}
\begin{itemize}
    \item \textbf{Wine}: \textit{S. cerevisiae} is the primary yeast used in winemaking, where the fermentation of sugars from grapes occurs in anaerobic conditions, favouring its growth and alcohol production \cite*{L9-ISAPP,L9-Coryn,LabManual}.
    \item \textbf{Beer}: In brewing, \textit{S. pastorianus} and \textit{S. cerevisiae} are both utilized, with the former favoured in lager production and the latter in ales. The presence of malt sugars provides a conducive environment for their growth \cite*{L9-ISAPP,L9-Coryn}.
    \item \textbf{Sourdough}: Sourdough employs \textit{S. exiguus} and \textit{C. milleri}, alongside \textit{S. cerevisiae}, utilizing the carbohydrates in flour under anaerobic conditions \cite*{L9-Coryn,L6-Yeasts}.
\end{itemize}

\subsubsubsection*{Potential yeast interactions with other microorganisms}
Saccharomyces spp. often interact with various microorganisms in fermentation environments:
\begin{itemize}
    \item \textbf{Bacterial Interactions}: In sourdough, \textit{S. cerevisiae} interacts with LAB, such as \textit{Lactobacillus}, which not only contributes to flavour but also enhances fermentation efficiency \cite*{L9-Coryn}.
    \item Bacterial Interactions: In sourdough, \textit{S. cerevisiae} interacts with LAB, such as \textit{Lactobacillus}, which not only contributes to flavour but also enhances fermentation efficiency \cite*{LabManual}.
\end{itemize}

\subsection{Bacillus}
\subsubsection*{Question}
Identify the most commonly occurring species of aerobic, endospore-forming microorganisms found in alkaline fermented foods. Mention at least 2-3 examples of desired technological properties of Bacillus spp. and how these technological properties influence the quality of the final product. Discuss shortly the implications of spontaneous as compared to controlled fermentations for the quality and safety of the final product.  

\subsubsection*{Keywords}
Table \ref{tab:KW-Bacillus} shows the keywords for this question and reflects the answer written in the following subsection.
\begin{table}[h]
    \centering
    \caption{Keywords for Bacillus}
    \label{tab:KW-Bacillus}
    \begin{tabular}{l|l}
        \textbf{Topic} & \textbf{Keywords} \\
        \hline
        Common AEB Species & \textit{Bacillus subtilis}, \textit{Bacillus licheniformis}, \textit{Bacillus cereus} and \textit{Lysinibacillus macroides}\\
        Technological Properties & Proteolysis, flavour development, umami taste and protein degradation\\
    \end{tabular}
\end{table}


\subsubsection*{Answer}
\subsubsubsection*{Commonly occurring species of aerobic, endospore-forming microorganisms in alkaline fermented foods}
Some commonly occurring species of aerobic, endospore-forming microorganisms (AEB) in alkaline fermented foods include \cite*{L8-ImpQuorum}:
\begin{itemize}
    \item \textit{Bacillus subtilis}
    \item \textit{Bacillus licheniformis}
    \item \textit{Bacillus cereus}
    \item \textit{Lysinibacillus macroides}
\end{itemize}

\subsubsubsection*{Desired technological properties of \textit{Bacillus} spp. and their influence on product quality}
\textbf{Proteolycity}: 
\begin{itemize}
    \item \textit{Bacillus} species possess significant proteolytic activity, which aids in protein degradation during fermentation. This property enhances flavour development by releasing amino acids and peptides that contribute to the umami taste \cite*{L8-ImpQuorum}.
\end{itemize}

\subsection{LAB-1}
\subsubsection*{Question}
The former \textit{Lactobacillus} genus has been split into 25 new genera. Among these new genera are \textit{Companilactobacillus}, \textit{Lactiplantibacillus}, \textit{Lactobacillus}, \textit{Lacticaseibacillus}, \textit{Levilactobacillus}, \textit{Lentilactobacillus}, and \textit{Limosilactobacillus}. From these genera, mention two species of relevance for fermentation of food or beverages, one of them being a specialist and the other a generalist; describe their habitat, typical phenotypes, and their contribution to the fermentation. Describe how you would identify the species by use of phenotypic and molecular methodologies.

\subsubsection*{Keywords}
Table \ref{tab:KW-LAB1} shows the keywords for this question and reflects the answer written in the following subsection.
\begin{table}[h]
    \centering
    \caption{Keywords for LAB-1} 
    \label{tab:KW-LAB1}
    \begin{tabularx}{\textwidth}{l|X}
        \textbf{Topic} & \textbf{Keywords} \\
        \hline
        Specialist Species & \textit{Lactobacillus acidophilus}, lactose, acidic environment, yogurt, probiotic \\
        Other Specialists & \textit{L. crispatus}, \textit{L. gasseri}, \textit{L. iners}, niche specialization \\
        Generalist Species & \textit{Lacticaseibacillus casei}, versatile, adaptable, lactic acid, dairy, plant-based fermentations \\
        Identification Methods & Gram staining, catalase test, 16S rRNA, MALDI-TOF \\
    \end{tabularx}
\end{table} 

\subsubsection*{Answer}
\subsubsubsection*{Specialis}
Need a specific environment, lactose and an acidic environment.
\begin{itemize}
    \item \textbf{Species}: \textit{Lactobacillus acidophilus}
    \item \textbf{Habitat}: Primarily found in the human gastrointestinal tract, especially in the small intestine, as well as in certain fermented dairy products.
    \item \textbf{Typical Phenotypes}: Gram-positive, rod-shaped, acid-tolerant with specific nutritional requirements such as lactose, making it suitable for nutrient-rich environments with low pH.
    \item \textbf{Contribution to Fermentation}: \textit{L. acidophilus} plays an important role in the fermentation of dairy products like yogurt. It produces lactic acid, which lowers the pH, giving the product a tangy flavour and extending its shelf life by inhibiting spoilage organisms. Its presence also adds probiotic benefits that support gut health, making it particularly valuable in dairy fermentation.
    \item \textbf{Identification}: Phenotypic tests such as Gram staining, catalase test, and microscope analysis can be used to identify \textit{L. acidophilus}. Molecular methods like 16S rRNA gene sequencing can provide more accurate species identification.
\end{itemize}

\vspace*{0.5em}
Other specialist:
\textit{Lactobacillus} species such as \textit{L. crispatus}, \textit{L. gasseri}, \textit{L. iners}, and \textit{L. jenseni} are commonly detected in the human vagina. Their niche specialization has
led to smaller genomes and has reinforced the very narrow ecological distribution of these species \cite*{L3-LAB}.

\subsubsubsection*{Generalist}
\begin{itemize}
    \item \textbf{Species}: \textit{Lacticaseibacillus casei}
    \item \textbf{Habitat}: Found in the human gastrointestinal tract, dairy products, and various plant-based fermentations.
    \item \textbf{Typical Phenotypes}: Gram-positive, rod-shaped, versatile, and highly adaptable to different pH levels and nutritional conditions, enabling it to survive in a range of fermentation environments.
    \item \textbf{Contribution to Fermentation}: \textit{L. casei} contributes to the fermentation of dairy and plant products by producing lactic acid, which lowers the pH, improves preservation, and adds a tangy flavour. Its metabolic flexibility allows it to thrive in both dairy and non-dairy fermentations, making it valuable for consistent product quality and safety across various food types.
    \item \item \textbf{Identification}: Phenotypic tests such as Gram staining, catalase test, and microscope analysis can be used to identify \textit{L. acidophilus}. Molecular methods like 16S rRNA gene sequencing can provide more accurate species identification.
\end{itemize}

\vspace*{0.5em}
Rapid identification of bacteria can be performed by whole-cell matrix-assisted laser desorption ionization-time off light (MALDI-TOF)methodology.The identification process is based on fingerprinting analyses of ribosomal proteins and other abundant basic proteins \cite*{L9-Coryn}.

\subsection{LAB-2}
\subsubsection*{Question}
Explain the strain definition for bacteria. Describe the methods you will use for strain typing and the principles behind these methods (select 2-3 methods). Additionally, explain how these methods can be used to determine whether different strains of \textit{Lacticaseibacillus casei} used as ripening cultures during cheese production are able to grow in the cheese.

\subsubsection*{Keywords}
Table \ref{tab:KW-LAB2} shows the keywords for this question and reflects the answer written in the following subsection.
\begin{table}[h]
    \centering
    \caption{Keywords for LAB-2} 
    \label{tab:KW-LAB2}
    \begin{tabular}{l|l}
        \textbf{Topic} & \textbf{Keywords} \\
        \hline
        Strain Definition & Genetic variant, phenotypic, genotypic, ANI, type strain \\
        Strain Importance & Food fermentation, epidemiology, behaviour distinction \\
        Strain Typing Methods & Sequencing, ON-Rep-Seq, Rep-PCR, PFGE \\
        Typing Principles & CDNA sequence analysis, repetitive sequences, banding patterns \\
    \end{tabular}
\end{table}

\subsubsection*{Answer}
\subsubsubsection*{Strain Definition for Bacteria}
A "strain" refers to a genetic variant or subtype of a microorganism, particularly bacteria. It is defined as an isolate or a group of isolates that can be distinguished from other isolates of the same genus and species based on specific characteristics \cite*{LS08}.
\begin{itemize}
    \item \textbf{Strain Definition for Bacteria}:
    \subitem A strain is often identified through a combination of phenotypic (observable characteristics) and genotypic (genetic information) traits. Common phenotypic characteristics include carbohydrate fermentation profiles and protein expression profiles \cite*{LS09}.
    \subitem Genotypically, strains can be differentiated using methods such as Average Nucleotide Identity (ANI), which may exceed 99\% for closely related strains, as well as various molecular typing techniques \cite*{LS09}.
    \item \textbf{Type and Reference Strains}:
    \subitem The term "type strain" refers to the strain on which the description of a species is based. A "reference strain" is a strain that is widely used within the scientific community and is often maintained in curated culture collections \cite*{LS09}.
    \item \textbf{Importance of Strain Differentiation}:
    \subitem Strain differentiation is essential in fields such as food fermentation and epidemiology, where specific strains can have distinct behaviors or effects \cite*{LS09}. 
    \subitem The ability to classify bacteria into strains allows researchers and practitioners to understand their applications better and their implications in various environments, such as in food products \cite{LS09}.
\end{itemize}

\subsubsubsection*{Methods for Strain Typing}
Detection of LAB strains from foods can be done by using the following molecular identification and characterization methods \cite*{LS05}:
\begin{itemize}
    \item \textbf{Sequencing}: DNA sequence analysis for typing.
    \item \textbf{ON-Rep-Seq}: Whole genome sequencing technique.
    \item \textbf{Rep-PCR}: Amplifies repetitive DNA for typing.
    \item \textbf{PFGE}: DNA separation using pulsed electric fields.
\end{itemize}

\subsubsubsection*{Principles Behind Strain Typing Methods}
This is a brief overview of the principles behind the methods used for strain typing \cite*{LS09,L3-SeqBasedClass}:
\begin{itemize}
    \item \textbf{Sequencing}: Involves sequencing specific genes or regions of the genome to identify genetic variations between strains. The sequences are compared to reference databases to determine strain identity and are coupled with e.g. Rep-PCR.
    \item \textbf{ON-Rep-Seq$^*$}: ON-Rep-Seq is a optimized version of the rep-PCR method, that enhances the capabilities of Rep-PCR, enabling consistent DNA amplification and the simultaneous processing of many samples efficiently.
    \item \textbf{Rep-PCR$^*$}: Primers target interspersed repetitive sequences. In the lab exercises we used (GTG)5 primers for rep-PCR. No prior knowledge of the DNA sequence is needed, and it is used for initial grouping of microorganisms to reduce sequencing costs.
    \item \textbf{PFGE$^{**}$}: Switching the current facilitate separation of large DNA fragments. The DNA fragments are separated based on their size, allowing for the comparison of the banding patterns between strains.
    \item $^*$Sequencing methods, $^{**}$PCR-based methods.
\end{itemize}

\vspace{0.5em}
Use the above mentioned methods to determine which strains are dominant during different maturation steps while the cheese is ripening. This can be start, mid, end or more steps. Compare the bands of the different strains to determine their presence in the cheese.

\subsection{LAB}
\subsubsection*{Question}
Give an overview of the antimicrobial compounds produced by LAB. How do you detect the inhibitory compounds and how do you investigate which types of inhibitory compounds are produced? How will you determine that an inhibitory compound is a bacteriocin? Classify bacteriocins and give examples of their inhibition spectrum and mode of action. 

\subsubsection*{Answer}
Pages used from \cite*{L3-LAB} ((pages 101, 153-156)) and \cite*{L8-MicroInFood}.
\subsubsubsection*{Overview of Antimicrobial Compounds Produced by LAB}
LAB produce a variety of antimicrobial compounds that are key to their role in food preservation. These include lactic acid, diacetyl, hydrogen peroxide, and bacteriocins \cite*{L3-LAB}. Bacteriocins, in particular, are ribosomally synthesized peptides that inhibit or kill closely related bacterial strains, and are known to be heat-stable. These compounds not only contribute to competitive inhibition of spoilage organisms but also enhance the safety of fermented foods. \cite*{L3-LAB}.

\subsubsubsection*{Detection of Inhibitory Compounds}
The detection of antimicrobial compounds, such as bacteriocins, typically involves culturing LAB and testing their supernatants for antimicrobial activity against indicator strains. The use of agar diffusion assays is a common method, where zones of inhibition around the LAB cultures indicate antimicrobial activity \cite*{L3-LAB}. To specifically detect bacteriocins, the inhibitory activity is tested for sensitivity to proteolytic enzymes such as protease and $\alpha$-chymotrypsin. If the activity is lost upon enzyme treatment, this indicates the presence of a proteinaceous inhibitory compound, which could be a bacteriocin \cite*{L3-LAB}

\subsubsubsection*{Investigating Types of Inhibitory Compounds}
To identify the type of inhibitory compound produced by LAB, different biochemical and molecular methods are used. Chromatographic techniques like HPLC, coupled with mass spectrometry, can be employed to characterize the molecular structure of the inhibitory compound. Genetic analysis of LAB strains can reveal the presence of bacteriocin gene clusters, which are typically organized in operons and encode both the bacteriocin peptide and its associated immunity proteins \cite*{L3-LAB}.

\subsubsubsection*{Determining if an Inhibitory Compound is a Bacteriocin}
An inhibitory compound is classified as a bacteriocin if it meets the following criteria: it is a ribosomally synthesized peptide, it exhibits antimicrobial activity that is proteinaceous (sensitive to protease treatment), and it targets bacteria \cite*{L3-LAB}. Additionally, bacteriocins are often associated with a specific immunity system that protects the producing strain from its own bacteriocin \cite*{L3-LAB}.

\subsubsubsection*{Classification of Bacteriocins}
Bacteriocins produced by LAB are categorized into different classes based on their structure and mode of action:
\begin{itemize}
    \item \textbf{Class I}: Small lantibiotics, such as nisin, which contain unusual amino acids like lanthionine. These attack cell membranes and are heat-stable \cite*{L3-LAB}.
    \item \textbf{Class II}: Small, heat-stable, non-lantibiotic bacteriocins. These attack cell membranes. These include \cite*{L3-LAB}:
    \subitem \textbf{Class IIa}: Pediocin-like bacteriocins, with strong anti-listerial activity
    \subitem \textbf{Class IIb}: Two-peptide bacteriocins
    \item \textbf{Class III}: Large, heat-labile proteins \cite*{L8-MicroInFood}.
    \item \textbf{Class IV}: Complex bacteriocins with additional glyco- or lipid moieties \cite*{L8-MicroInFood}.
\end{itemize}

\subsubsubsection*{Inhibition Spectrum and Mode of Action}
The inhibition spectrum of bacteriocins can vary. For example, nisin is effective against a broad range of Gram-positive bacteria, including \textit{Listeria monocytogenes}. The mode of action of bacteriocins typically involves binding to specific receptors on the target cell membrane, leading to pore formation and cell death \cite*{L3-LAB}. Some bacteriocins, like the lantibiotics, bind to lipid II in the bacterial cell wall, disrupting cell wall synthesis. Others, like Class IIa bacteriocins, form pores in the membrane, leading to leakage of cytoplasmic contents and eventual cell lysis \cite*{L3-LAB}.

\subsection{Moulds}
\subsubsection*{Question}
Mention 2-3 mould genera and methodologies for their identification to species level. Give examples (2-3) of how moulds are used in fermentation of foods and beverages. Discuss the positive and negative characteristics of moulds in food production. 

\subsubsection*{Answer}
\subsubsubsection*{Mould Generas}
Common mould genera relevant to food production include \textit{Penicillium}, \textit{Aspergillus}, and \textit{Rhizopus}. These genera are significant for both food fermentation and spoilage.
\begin{itemize}
    \item \textit{Penicillium}: This genus includes species like \textit{Penicillium roqueforti}, used in blue cheese fermentation \cite*{L8-MicroInFood}.
    \item \textit{Aspergillus}: Species like \textit{Aspergillus oryzae} are utilized in the production of soy sauce and sake \cite*{L1-DiversityMicro}.
    \item \textit{Rhizopus}: Used in traditional Asian fermented foods such as tempeh \cite*{L1-DiversityMicro,LS03}.
\end{itemize}

\subsubsubsection*{Identification Methodologies}
Molecular techniques are key for species-level identification of moulds. Techniques include:
\begin{itemize}
    \item \textbf{Micro-/macro-morphological methods}: The characteristics of each species vary, thus analysing the respective moulds micro-/macro-morphological characteristics can give an indication of the respective species \cite*{LabManual}.
    \item \textbf{PCR-based methods}: Widely applied for species-specific identification through DNA amplification \cite*{L1-DiversityMicro}.
    \item \textbf{MALDI-TOF mass spectrometry}: This technique allows for fast, cost-effective identification by comparing protein profiles \cite*{L9-Coryn}.
    \item \textbf{Sequencing}: Specifically, sequencing ribosomal RNA genes (such as 18S/26S rRNA) is a gold standard for species identification of moulds \cite*{L3-SeqBasedClass}.
\end{itemize}

\subsubsubsection*{Moulds in Fermentation of Foods and Beverages}
\begin{itemize}
    \item \textbf{Cheese Production}: Penicillium roqueforti is used in blue-veined cheeses, contributing to flavour and texture development \cite*{L8-MicroInFood}.
    \item \textbf{Soy Sauce and Sake}: \textit{Aspergillus oryzae} plays a crucial role in breaking down starches in grains, enhancing the fermentation process \cite*{L1-DiversityMicro}.
    \item \textbf{Tempeh Production}: \textit{Rhizopus} species are essential for tempeh fermentation, where they ferment soybeans into the protein-rich product tempeh \cite*{L1-DiversityMicro}.
\end{itemize}

\subsubsubsection*{Positive and Negative Characteristics of Moulds in Food Production}
\textbf{Positive Characteristics}:
\begin{itemize}
    \item \textbf{Flavour Development}: Moulds contribute to the unique flavours in foods such as cheese and fermented soy products \cite*{LS03}.
    \item \textbf{Nutritional Enhancement}: Mould fermentation can increase the digestibility of nutrients, particularly in products like tempeh. Also enzymatic powers helped by amylase, protease and lipases help enhancing the nutritional value \cite*{L1-DiversityMicro,LS03}.
    \item \textbf{Bio-control}: Some moulds have anti-fungal properties, which can help prevent product spoilage by harmful microorganisms \cite*{L1-DiversityMicro}.
\end{itemize}

\vspace{0.5em}  
\textbf{Negative Characteristics}:
\begin{itemize}
    \item \textit{Toxin Production}: Certain species of moulds produce toxins, which are harmful to human health. Moreover, a moulds toxin production can change depending on the environments, thus its imperative to study the change before using a mould in a new environments \cite*{LS03}.
    \item \textbf{Food Spoilage}: Moulds such as Aspergillus can spoil stored food products, leading to significant economic losses \cite*{L1-DiversityMicro}
\end{itemize}

\subsection{Microbial Interactions/Quorum Sensing}
\subsubsection*{Question}
Describe the principles behind quorum sensing (QS) and give examples of microbial properties (traits) that are under QS control. Explain how QS differs from other types of microbial interactions. Mention 2-3 types of microbial interactions which are not under QS control and explain their mode of action.  

\subsubsection*{Answer}
\subsubsubsection*{Principles Behind Quorum Sensing (QS)}
Quorum sensing (QS) is a cell-density dependent mechanism where microorganisms communicate via signalling molecules to regulate gene expression once a critical population threshold is reached. This allows for synchronized behaviors across the microbial population \cite*{L8-ImpQuorum}.

\subsubsubsection*{Microbial Properties Under QS Control}
Microbial traits under QS control in food-related microorganisms include biofilm formation, acid stress tolerance, bacteriocin production, competence, adhesion, morphological switches, and growth orientation \cite*{L8-ImpQuorum}.

\subsubsubsection*{Differences from Other Microbial Interactions}
Quorum sensing involves the production and detection of signalling molecules by microorganisms, allowing them to alter behaviour in response to cell population density. This differs from other microbial interactions like nutrient competition or production of inhibitory compounds, which do not rely on such communication mechanisms \cite*{L8-ImpQuorum}.

\subsubsubsection*{Examples of microbial interactions not under QS control}
\begin{itemize}
    \item \textbf{Nutrient Competition}: Microorganisms compete for limited resources such as nutrients, impacting their growth and survival in environments like fermented foods \cite*{L8-ImpQuorum}.
    \item \textbf{Production of Organic Acids}: The production of organic acids by lactic acid bacteria (LAB) during fermentation reduces the pH, inhibiting the growth of spoilage microorganisms \cite*{L8-ImpQuorum}.
    \item \textbf{Bacteriocin Production}: Bacteriocins are antimicrobial peptides produced by certain bacteria, including LAB, which inhibit closely related species and contribute to food preservation \cite*{L8-ImpQuorum}.
\end{itemize}

\subsection{Probiotics}
\subsubsection*{Question}
Give a definition of probiotics and name 2-3 species where strains are commonly applied as probiotics. Give examples on properties that a potential probiotic candidate can fulfil. Line out diverse ways of investigating these probiotic properties. Probiotics might have beneficial effects against diarrhea. What are the proposed mechanisms behind this effect?

\subsubsection*{Answer}
\subsubsubsection*{Definition of Probiotics}
The consensus definition of probiotics is: “live microorganisms that, when administered in adequate amounts, confer a health benefit on the host” \cite*{L10-Pro_Pre}.

\subsubsubsection*{Common Probiotic Species}
These are species with claimed probiotic effect, see lecture 19 for more examples \cite*{LS19}:
\begin{itemize}
    \item \textit{Lactobacillus acidophilus} (often included in fermented dairy products like yogurt).
    \item \textit{Bifidobacterium animalis} (modulates gut microbiota and supporting immune function).
    \item \textit{Saccharomyces boulardii} (particularly effective in reducing diarrhea and preventing gastrointestinal disorders).
\end{itemize}

\subsubsubsection*{Properties of Potential Probiotic Candidates}
Probiotic microorganisms act via a variety of means, in the following list, some propersites are listed \cite*{L10-Pro_Pre}:
\begin{itemize}
    \item Modulation of immune function
    \item Production of organic acids and antimicrobial compounds
    \item Interaction with resident microbiota
    \item Interfacing with the host
    \item Improving gut barrier integrity
    \item Enzyme formation
    \item Can increase phagocytosis or natural killer cell activity and interact directly with dendritic cells
\end{itemize}

\subsubsubsection*{Investigating Probiotic Properties}
Much the derived knowledge on probiotic mechanisms is found while researching animal cell cultures in vitro, or on human models ex vivo. Weather one or the other methods has been used, it is imperative that the probiotics must be extensively evaluated and shown to have positive health effects \cite*{L10-Pro_Pre,LS19}.

\subsubsubsection*{Mechanisms Against Diarrhea}
The following effects can have a positive impact on diarrhea \cite*{L10-Pro_Pre}:
\begin{itemize}
    \item Increasing short-chain fatty acids (SCFAs) and hydration in the colon.
    \item The ability to down-regulate inflammatory mediators and increase epithelial barrier function.
    \item Reduce diarrhea incidence by producing organic acids, lowering gut pH, and inhibiting pathogens
\end{itemize}

\subsection{Plant-based Fermented Foods/Health}
\subsubsection*{Question}
Describe how microorganisms, through fermentation of plant-based foods, can benefit nutritional quality and human health. Include at least 2-3 different effects and discuss their underlying mechanisms. 

\subsubsection*{Answer}
\begin{itemize}
    \item \textbf{Nutrient Bioavailability}: Fermentation increases the bioavailability of vitamins and minerals. For instance, tempeh fermentation enhances vitamin $B_{12}$ production and amino acid availability by breaking down proteins into smaller peptides and amino acids, improving digestibility and nutrient absorption \cite*{LS22}.
    \item \textbf{Reduction of Anti-Nutritional Factors}: Microorganisms such as \textit{Lactiplantibacillus plantarum} degrade anti-nutritional factors like phytic acid, raffinose, and stachyose in plant-based foods. This process increases the bioavailability of minerals like calcium and iron, which are otherwise bound by these compounds \cite*{LS22}.
    \item \textbf{Production of Health-Promoting Metabolites}: LAB and other microorganisms produce beneficial metabolites such as short-chain fatty acids (e.g., butyrate which lowers blood pressure), which enhance gut health by maintaining the integrity of the gut barrier and reducing the risk of metabolic disorders \cite*{LS22,LS19}.
\end{itemize}

\subsubsubsection*{Kimchi Fermentation}
\textbf{Pathways Involved}: Kimchi fermentation primarily involves LAB, including species such as \textit{Leuconostoc mesenteroides}, \textit{Lactobacillus plantarum}, and \textit{Lactobacillus sakei}. These microorganisms metabolize sugars into lactic acid, which lowers the pH and inhibits the growth of harmful pathogens \cite*{LS14}.

\textbf{Health benefits}:
\begin{itemize}
    \item LAB in kimchi produce lactic acid, which creates an acidic environment that preserves the food and promotes gut health \cite*{LS14}
    \item Fermentation enhances the bioavailability of certain nutrients, including B vitamins, which are important for overall metabolic health \cite*{LS19}.
\end{itemize}

\subsubsubsection*{Cacao Fermentation}
\textbf{Pathways Involved}: Cacaos successive fermentation is driven by yeasts, LAB, and AAB. Yeasts ferment sugars into ethanol, followed by LAB converting sugars and organic acids into lactic acid, and finally AAB oxidize ethanol into acetic acid, which contributes to flavour development \cite*{ORM_1}.

\textbf{Health benefits}:
\begin{itemize}
    \item Fermentation enhances the levels of bioactive compounds such as flavonoids and polyphenols, which have antioxidant properties and can reduce the risk of cardiovascular diseases \cite*{ORM_1}.
    \item The high content of monomeric (epicatechin and catechin) and oligomeric (procyanidins) flavanols has through studies shown positive health benefits \cite*{ORM_1}.
\end{itemize}

\subsubsubsection*{Koji Fermentation}
\textbf{Pathways Involved}: Koji fermentation involves the mould \textit{Aspergillus oryzae}, which produces enzymes like amylases and proteases that break down starches into sugars and proteins into amino acids. This saccharification process is essential for producing products like miso, soy sauce, and sake \cite*{LS11}.

\textbf{Health benefits}:
Koji fermentation enhances the digestibility of proteins and produces umami compounds such as glutamate, which contribute to flavour and increased nutrient absorption aids in the production of vitamins, such as vitamin $B_2$ and $B_{12}$, which are essential for maintaining energy levels and neurological health \cite*{LS11}.

\subsection{Sequencing Methods}
\subsubsection*{Question}
Define culture-independent and culture-dependent techniques for microbial characterization of fermented food and describe advantages and disadvantages of the two approaches. Give 1-2 examples on how you can apply Nanopore amplicon sequencing for species identification. Discuss the advantages and disadvantages of amplicon sequencing compared to whole genome sequencing. 

\subsubsection*{Answer}
\subsubsubsection*{Culture-Independent Techniques}
Culture-independent methods bypass the need for growing microorganisms in a lab. These techniques include high-throughput sequencing (HTS), such as 16S rRNA sequencing, which allows for the identification of bacterial species directly from environmental samples by analysing their genetic material.

\textbf{Advantages}:
\begin{itemize}
    \item They allow for the detection of both culture-able and non-culture-able organisms, which is especially useful in complex microbial communities such as fermented foods \cite*{L3-SeqBasedClass,L5-HighThroughput}.
    \item Faster results compared to culture-dependent methods, as they do not require incubation periods \cite*{L5-DNAEnrichment}.
    \item Provide insights into the diversity and structure of microbial communities through sequencing techniques such as 16S rRNA sequencing \cite*{L3-SeqBasedClass}.
\end{itemize}

\textbf{Disadvantages}:
\begin{itemize}
    \item Does not offer functional data about the microorganisms, such as metabolic activity \cite*{L5-HighThroughput,L3-SeqBasedClass}.
    \item Some techniques (e.g., 16S rRNA sequencing) may not resolve closely related species \cite*{L5-HighThroughput}.
\end{itemize}

\subsubsubsection*{Culture-Dependent Techniques}
Culture-dependent methods involve the isolation and cultivation of microorganisms on specific media before further testing. These methods are essential for obtaining live cultures for functional and strain-level analyses.

\textbf{Advantages}:
\begin{itemize}
    \item Allow for more detailed studies of bacterial phenotypes and strain-specific characteristics, especially when combined with techniques such as multilocus sequence typing (MLST) and pulsed-field gel electrophoresis (PFGE) \cite*{L5-DNAEnrichment}
    \item Enable researchers to conduct functional assays, such as antibiotic resistance testing or metabolic profiling, on live cultures \cite*{L3-SeqBasedClass}.
\end{itemize}

\textbf{Disadvantages}:
\begin{itemize}
    \item Limited to organisms that can grow under specific lab conditions, excluding non-culture-able organisms \cite*{L5-HighThroughput}.
    \item Time-consuming and labor-intensive, as it requires the incubation of individual colonies on culture-specific media \cite*{L5-DNAEnrichment}.
\end{itemize}

\subsubsubsection*{Nanopore Amplicon Sequencing for Species Identification}
Nanopore amplicon sequencing is a culture-independent technique which provides real time sequencing of long DNA fragments. Moreover, Nanopore allows the user to identify species and strains without the need for fluorescence chemistry. 
Nanopores are small holes created by proteins. The Nanopore sequencing principles is detection by the generated current of the nucleotide which passes through the Nanopore. The changes in the current are characteristic of a particular nucleotide base \cite*{LS10,L3-SeqBasedClass}.

\textbf{Examples of Applications}:
\begin{itemize}
    \item Nanopore sequencing can be applied for the identification of \textit{Salmonella enterica} at the strain level using repetitive extragenic palindromic PCR (Rep-PCR) coupled with ON-rep-seq \cite*{L5-DNAEnrichment}.
    \item Nanopore sequencing can also be applied to rapidly sequence the 16S rRNA gene from mixed microbial samples, allowing species identification in fermented food samples \cite*{L3-SeqBasedClass}.
\end{itemize}


\subsubsubsection*{Amplicon Sequencing vs. Whole Genome Sequencing}
\textbf{Advantages of Amplicon Sequencing}:

\begin{itemize}
    \item Long-read sequencing allows for the identification of full-length 16S rRNA gene sequences, providing more accurate species identification compared to short-read sequencing \cite*{LS10}.
    \item Fast runtime allows for multiple runs in a short amount of time \cite*{LS10}.
\end{itemize}

\textbf{Disadvantages}:
\begin{itemize}
    \item Higher error rates (38\% in 2015, 5\% in 2016) compared to short-read sequencing technologies like Illumina (1\%) \cite*{LS10}.
    \item Limited to specific genomic regions, making it less informative for strain-level differentiation and functional genomics compared to whole-genome sequencing \cite*{L5-DNAEnrichment}.
\end{itemize}

\textbf{Advantages of Whole Genome Sequencing}:
\begin{itemize}
    \item Provides comprehensive genomic information, including strain-level differences, gene content, and functional pathways \cite*{L5-DNAEnrichment}.
\end{itemize}

\textbf{Disadvantages of Whole Genome Sequencing}:
\begin{itemize}
    \item More expensive and computationally demanding than amplicon sequencing \cite*{L5-DNAEnrichment,L3-SeqBasedClass}.
\end{itemize}






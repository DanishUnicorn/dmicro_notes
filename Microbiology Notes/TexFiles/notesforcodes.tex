
\section{Rules of probability}

\subsection{Probability density function}

The probability density function (pdf) of a random variable $X$ is a function $f(x)$ that describes the likelihood of observing a value $x$ of $X$.

\begin{align}
    f(x) = P(X=x), \quad 
    \text{where } x \in \text{set of possible values of } X.
\end{align}

It does not make sense to talk about the pdf at a specific value for a continuous random variable (as this will always be zero). Instead, we talk about the probability of observing a value in an interval.

\subsection{Cumulative density function}

The cumulative density function (cdf) of a random variable $X$ is a function $F(x)$ that describes the probability of observing a value less than or equal to $x$.

\begin{equation}
    F(x) = P(X \leq x)  
    \quad \text{where } x \in \text{set of possible values of } X.
\end{equation}

\subsection{Link between CDF and PDF}

The pdf and cdf are linked by the following equation

\begin{equation}
    p(a \leq X \leq b) = F(b) - F(a),
\end{equation}

where $b > a$. 

\subsection{Complement rule}

The probability of an event not happening is one minus the probability of the event happening. 

\begin{equation}
    p(\bar{A}) = 1 - p(A)
\end{equation}

This can be used in the following way:

\begin{equation}
    p(X > x) = 1 - p(X \leq x) = 1 - F(x)
\end{equation}
    
\section{Rules of random variables}

\subsection{Mean rule}

Let $X$ be a random variable and $a$ and $b$ be constants $\in \R$ then

\begin{equation}
    \E(aX + b) = a\E(X) + b.
\end{equation}

\subsection{Variance rule}

Let $X$ be a random variable and $a$ and $b$ be constants $\in \R$ then

\begin{equation}
    \text{Var}(aX + b) = a^2 \text{Var}(X).
\end{equation}

\subsection{Mean of linear combinations}

Let $X_1, ... , X_n$ be \textcolor{red}{independent} random variables then

\begin{equation}
    \E(aX_1 + bX_2 + ... + cX_n) = a\E(X_1) + b\E(X_2) + ... + c\E(X_n).
\end{equation}

\subsection{Variance of linear combinations}

Let $X_1, ... , X_n$ be \textcolor{red}{independent} random variables then

\begin{equation}
    \text{Var}(aX_1 + bX_2 + ... + cX_n) = a^2\text{Var}(X_1) + b^2\text{Var}(X_2) + ... + c^2\text{Var}(X_n).
\end{equation}

\section{Discrete distributions}


\subsection{Binomial distribution}

Let the random variable $X$ be the number of successes in independent $n$ draws with replacement. Then $X$ follows the binomial distribution

\begin{equation}
X \sim B(n,p),
\end{equation}

where $n$ is the number of draws and $p$ is the probability of success in each trial. The binomial pdf describes the probability of obtaining $x$ successes in $n$ draws

\begin{equation}
f(x;n,p) = P(X=x) = \binom{n}{x} p^x (1-p)^{n-x},
\end{equation}

where,

\begin{equation}
\binom{n}{x} = \frac{n!}{x!(n-x)!},
\end{equation}

is the number of distinct sets of $x$ elements which can be chosen from a set of $n$ elements. Remember that $n! = n \cdot (n-1) \cdot (n-2) \cdot ... \cdot 2 \cdot 1$.

The mean of a binomial distributed random variable is

\begin{equation}
\mu = np,
\end{equation}
and the variance is,
\begin{equation}
\sigma^2 = np(1-p).
\end{equation}


\begin{highlight}
    The binomial distribution has the following assumptions:
\begin{itemize}
    \item The draws are independent.
    \item Each draw is made with replacement. That is, the probability of success does not change from draw to draw.
    \item If the set from where the "draws" are taken is approximately infinite, the binomial distribution can be used without replacement. 
\end{itemize}
\end{highlight}


\subsection{Hypergeometric distribution}

Let the random variable $X$ be the number of successes in $n$ draws without replacement. Then $X$ follows the hypergeometric distribution

\begin{equation}
X \sim H(n, a, N),
\end{equation}

where $a$ is the number of successes in the population and $N$ is the population size. The hypergeometric pdf describes the probability of obtaining $x$ successes in $n$ draws

\begin{equation}
f(x;n,a,N) = P(X=x) = \frac{\binom{a}{x} \binom{N-a}{n-x}}{\binom{N}{n}},
\end{equation}

where,

\begin{equation}
\binom{a}{b} = \frac{a!}{b!(a-b)!},
\end{equation}

is the number of distinct sets of $b$ elements which can be chosen from a set of $a$ elements. 

The mean of a hypergeometric distributed random variable is

\begin{equation}
\mu = \frac{na}{N},
\end{equation}

and the variance is,

\begin{equation}
\sigma^2 = n\frac{a(N-a)}{N^2} \cdot \frac{N-n}{N-1}
\end{equation}

\subsection{Poisson distribution}

Let the random variable $X$ be the number of events in a given interval (time, space or other). Then $X$ follows the Poisson distribution

\begin{equation}
X \sim Po(\lambda),
\end{equation}

where $\lambda$ is the average number of events in the interval (called rate or intensity). The Poisson pdf describes the probability of observing $x$ events in an interval

\begin{equation}
f(x;\lambda) = P(X=x) = \frac{\lambda^x}{x!} e^{-\lambda},
\end{equation}

The mean of a Poisson distributed random variable is
\begin{equation}
\mu = \lambda,
\end{equation}
and the variance is,
\begin{equation}
\sigma^2 = \lambda.
\end{equation}

\subsubsection{Rate scaling of Poisson distribution}

The Poisson distribution is often used to model the number of events in a given interval. If the interval is scaled by a factor $c$, the rate of events will also scale by $c$. This means that if $X \sim Po(\lambda)$, then $cX \sim Po(c\lambda)$.

\begin{example}{Rate scaling}{rate_scaling}
If a 10 mL sample has 5 bacteria on average,
\begin{equation*}
X \sim Po(\lambda = 5),
\end{equation*}

Then we can find the rate for a 20 mL sample by scaling the rate by the factor $c = 2$,

\begin{equation*}
cX \sim Po(c\lambda) = Po(2 \cdot 5) = Po(10).
\end{equation*}
\end{example}

\section{Color codes for highlighting}
Apricot
Aquamarine
Bittersweet
BlueGreen
BrickRed
BurntOrange
CarnationPink
Cerulean
CornflowerBlue
Dandelion
Emerald
ForestGreen
Fuchsia
Goldenrod
Lavender
LimeGreen
Magenta
Melon
MidnightBlue
Mulberry
NavyBlue
OliveGreen
Orange
Orchid
Peach
Periwinkle
PineGreen
Plum
RawSienna
RedOrange
RoyalBlue
RubineRed
SeaGreen
SkyBlue
SpringGreen
Tan
TealBlue
Thistle
Turquoise
Violet
WildStrawberry
YellowGreen







\subsection{Article 2 - Diversity of Microorganisms in Global Fermented Foods and Beverages}    

\subsubsection*{Introduction}
The article describes how rice with fermented and non-fermented legumes is a staple diet in many counties in Asia. Wheat/barley-based breads/loaves followed bt milk and fermented milk products, meat and fermented meats are more common in the western world (West Asia, Europe and North America). The staple diet in Africa and South America comprise of sorghum and maize with wild legume seeds, meat and milk products \cite*{L1-DiversityMicro}. 
Consortia of microorganisms are found naturally in uncooked plant and animal materials, utensils, food containers, earthen pots and in general in the environment. When producing fermented foods, it is normal to introduce a starter culture, which contains functional microorganisms \cite*{L1-DiversityMicro}. The microorganisms in fermented food has converted the chemical composition of the raw food, which results in a change in the nutritional value of the food, making it more enriched \cite*{L1-DiversityMicro}.

\subsubsection*{Microorganisms in fermented foods}

\textbf{Lactic acid bacteria (LAB)} are one of the most important groups of microorganisms in fermented foods. They are used in the production of fermented foods and beverages. Some of the major genra of the LAB are listed below \cite*{L1-DiversityMicro}:
\begin{highlight}
    \begin{multicols}{3}
        \begin{itemize}
            \item \textit{Alkalibacterium}
            \item \textit{Carnobacterium}
            \item \textit{Enterococcus}
            \item \textit{Lactobacillus}
            \item \textit{Lactococcus}
            \item \textit{Leuconostoc}
            \item \textit{Oenococcus}
            \item \textit{Pediococcus}
            \item \textit{Streptococcus}
            \item \textit{Tetragenococcus}
            \item \textit{Vagococcus}
            \item \textit{Weissella}
        \end{itemize}
    \end{multicols}
\end{highlight}

\textbf{\textit{Bacillus}} has ben found in alkaline-fermented foods in Asia anf Africa. The species of \textit{Bacillus} that are found in abundance in legume-based fermented foods are listed bellow \cite*{L1-DiversityMicro}:

\begin{highlight}
    \begin{multicols}{3}
        \begin{itemize}
            \item \textit{Bacillus amyloliquefaciens}
            \item \textit{Bacillus circulans}
            \item \textit{Bacillus coagulans}
            \item \textit{Bacillus firmus}
            \item \textit{Bacillus licheniformis}
            \item \textit{Bacillus megaterium}
            \item \textit{Bacillus pumilus}
            \item \textit{Bacillus subtilis}
            \item \textit{Bacillus subtilis variety natto}
            \item \textit{Bacillus thuringiensis}            
        \end{itemize}
    \end{multicols}
\end{highlight}

There has been reported several species of \textit{Kocuria, Micrococcus} and \textit{Staphylococcus} in fermented foods. The species of \textit{Kocuria} are found in abundance in fermented milk products, fermented sausages, meat-, and fish products \cite*{L1-DiversityMicro}.

\vspace{1em}

\textbf{Yeasts} are also associated with fermentation of foods and alcoholic beverages. The yeasts named in the article are listed bellow \cite*{L1-DiversityMicro}:
\begin{highlight}
    \begin{multicols}{4}
        \begin{itemize}
            \item \textit{Brettanomyces}
            \item \textit{Candida}
            \item \textit{Cryptococcus}
            \item \textit{Debaryomyces}
            \item \textit{Dekkera}
            \item \textit{Galactomyces}
            \item \textit{Geotrichum}
            \item \textit{Hansenula}
            \item \textit{Hanseniaspora}
            \item \textit{Hyphopichia}
            \item \textit{Issatchenkia}
            \item \textit{Metschnikowia}
            \item \textit{Saccharomyces}
            \item \textit{Pichia}
            \item \textit{Kazachstania}
            \item \textit{Rhodotorula}
            \item \textit{Saccharomycodes}
            \item \textit{Kluyveromyces}
            \item \textit{Rhodosporidium}
            \item \textit{Saccharomycopsis}
            \item \textit{Schizosaccharomyces}
            \item \textit{Sporobolomyces}
            \item \textit{Torulaspora}
            \item \textit{Torulopsis}
            \item \textit{Trichosporon}
            \item \textit{Yarrowia}
            \item \textit{Zygosaccharomyces}                  
        \end{itemize}
    \end{multicols}
\end{highlight}

\textbf{Filamentous molds} are also found in fermented foods. They hold a major role in various fermented products in resprect of enzyme production and in the degradation of anti-nutritive factors. The listed filamentous molds are listed bellow \cite*{L1-DiversityMicro}. 

\begin{highlight}
    \begin{multicols}{4}
        \begin{itemize}
            \item \textit{Actinomucor}
            \item \textit{Amylomyces}
            \item \textit{Aspergillus}
            \item \textit{Monascus}
            \item \textit{Mucor}
            \item \textit{Neurospora}
            \item \textit{Paracilomyces}
            \item \textit{Penicillium}
            \item \textit{Rhizopus}
            \item \textit{Ustilago}
        \end{itemize}
    \end{multicols}
\end{highlight}

\subsubsection*{Taxonomic tools for identification of microorganisms from fermented foods}

A commonly used method for profiling both culturable and non-culturable microbial populations from fermented foods is DNA extraction. The technique is based on the separation of PCR-amplified DNA fragments by denaturing gradient gel electrophoresis (DGGE). This method is both used for profiling bacterial populations and yeast populations in fermented foods. Also, in fermented foods a combination of Propidium MonoAzide (PMA) treatment, which is done before DNA-extraction, and molecular quantifying methods can be used to enumerate the viable microorganisms accurately \cite*{L1-DiversityMicro}.
Random amplification of polymorphic DNA (RAPD) is a typing method based on the genomic DNA fragment profiles amplified by rep-PCR, and is commonly used for differentiation of LAB strains from fermented foods and other bacteria or yeast in the consortia. This technique is able to differentiate between strains of the same species \cite*{L1-DiversityMicro}.
Amplified fragment length polymorphism (AFLP) is another technique which is used to identify and discriminate LAB strains. The technique is based on the selective amplification and separation of genomic restriction fragments \cite*{L1-DiversityMicro}.

DGGE techniques and temperature gradient gel electrophoresis (TGGE) are used to study the microbial diversity in fermented foods. The techniques are based on the separation of PCR-amplified 16s rDNA and 26s rDNA gene fragments by denaturing gradient gel electrophoresis. The techniques are used to study the microbial diversity in fermented foods. DGGE is quite time consuming and is not able to determine relative abundance of the dominant species, nor is it able to distinguish between viable and non-viable cells \cite*{L1-DiversityMicro}.

Next generation sequencing (NGS) are effective tools for studying the microbial diversity in fermented foods. The techniques are able to sequence all the genetic material present in a sample, providing a comprehensive view of the microbial community's diversity and functional potential. These techniques require extensive training and a well-equipped molecular laboratory \cite*{L1-DiversityMicro}.

In table \ref{tab:NGS_techniques} some of the NGS which is listed in the text are described. These NGS are used to study the microbial diversity in fermented foods \cite*{L1-DiversityMicro}.


\begin{table}[htbp]
    \centering
    \caption{Some NGS techniques with their respective key-points}
    \label{tab:NGS_techniques}
    \begin{tabularx}{\textwidth}{p{0.33\textwidth}|p{0.33\textwidth}|p{0.33\textwidth}}
        \textbf{Metagenomics} & \textbf{Phylobiomics} & \textbf{Metatranscriptomics} \\
        \hline
        \begin{itemize}[left=0pt]
            \item Sequences all the genetic material present in a sample, providing a comprehensive view of the microbial community's diversity and functional potential.
            \item Allows the study of microbial communities without the need to culture individual species, overcoming limitations of traditional microbiology methods.
            \item By sequencing entire genomes, metagenomics can identify functional genes involved in processes like metabolism, antibiotic resistance, or nutrient cycling.
        \end{itemize} &
        \begin{itemize}[left=0pt]
            \item Is primarily concerned with analyzing the evolutionary relationships between microbial species, often using 16S rRNA or other conserved genes.
            \item It helps characterize the diversity and phylogenetic structure of microbial communities, determining how different species are related within an ecosystem.
            \item This technique is useful for identifying species and classifying them taxonomically based on their evolutionary lineage.
        \end{itemize} &
        \begin{itemize}[left=0pt]
            \item Focuses on sequencing the RNA transcripts present in a microbial community, providing a snapshot of active genes and gene expression levels.
            \item Reveals what genes are actively being expressed in a community at a given time.
            \item It helps assess how microbial communities respond to environmental changes, stress, or disease by analyzing shifts in gene expression.
        \end{itemize} \\
    \end{tabularx}
\end{table}

\subsubsection*{Global fermented foods}
A study reported approximately 3500 fermented food- and beverages, which were divided into 250 categories. Globally, fermented foods arae classified into 9 groups based of the substrates used from the plant- or animal source. These groups are listed in table \ref*{tab:Major_groups} \cite*{L1-DiversityMicro}.

\begin{table}[h]
    \centering
    \caption{The 9 groups of fermented foods and beverages}
    \label{tab:Major_groups}
    \begin{tabular}{c|>{\centering\arraybackslash}m{0.5\textwidth}}
        \textbf{Number} & \textbf{Group} \\
        \hline
        1 & Fermented cereals \\
        2 & Fermented vegetables and bamboo shoots \\
        3 & Fermented legumes \\
        4 & Fermented roots/tubers \\
        5 & Fermented milk products \\
        6 & Fermented and preserved meat products \\
        7 & Fermented, dried and smoked fish products \\
        8 & Miscellaneous fermented products \\
        9 & Alcoholic beverages \\
    \end{tabular}
\end{table}

\subsubsection*{Fermented milk products}
Fermented milk products can be classified into two main groups based on the microorganisms involved; Lactic fermentation and Fungal-lactic fermentations. The lactic fermentation is dominated by Lactic Acid Bacteria (LAB), including thermophilic- (e.g., yogurt, Bulgarian buttermilk), probiotic-  (e.g., acidophilus milk, bifidus milk), and mesophilic types  (e.g., natural fermented milk, cultured milk, cultured cream, cultured buttermilk). The Fungal-lactic fermentations involves cooperation between LAB and yeasts to produce products like alcoholic milks and moldy milks \cite*{L1-DiversityMicro}.

Natural fermentation is an ancient method of milk processing that uses raw or boiled milk to ferment spontaneously. Starter cultures are used to initiate fermentation and can be either primary or secondary. The primary cultures mostly consist of \textit{Lactococcus lactis}, \textit{Lactobacillus delbrueckii}, and \textit{Streptococcus thermophilus}, which participate in acidification. The secondary cultures are used in cheese-making and include \textit{Brevibacterium linens}, \textit{Propionibacterium freudenreichii}, and \textit{Penicillium camemberti}, which develop flavor and texture during ripening \cite*{L1-DiversityMicro}.

Non-Starter Lactic Acid Bacteria (NSLAB) are present in high numbers in fermented milk and include \textit{Enterococcus durans}, \textit{Lactobacillus casei}, and \textit{Staphylococcus} spp \cite*{L1-DiversityMicro}.

\subsubsection*{Fermented cereal foods}

Fermented cereal foods are prepared differently in various regions of the world. In Asia, rice is fermented to produce alcoholic beverages or food beverages using mixed cultures. In Europe, America, and Australia, cereals like wheat, rye, barley, and maize are fermented using natural fermentation or commercial baker's yeast. In Africa, fermented cereal foods are used as staples, complementary, and weaning foods for infants and young children \cite*{L1-DiversityMicro}.
In Europe, traditional methods of bread preparation are still practiced without using commercial baker's yeast. Yeasts and LAB conduct dough fermentation, resulting in sourdough breads with higher lactic acid and acetic acid content \cite*{L1-DiversityMicro}.
Some LAB as \textit{Enterococcus}, \textit{Lactococcus}, \textit{Lactobacillus}, \textit{Leuconostoc}, \textit{Pediococcus}, \textit{Streptococcus}, and \textit{Weissella} are deeply involved in cereal fermentation. Also some yeasts have a significant role such as the strains of \textit{Saccharomyces cerevisiae} are the principal yeast, but other non-\textit{Saccharomyces} yeasts like \textit{Candida}, \textit{Debaryomyces}, and \textit{Hansenula} is also found in cereal fermentation \cite*{L1-DiversityMicro}.

\subsubsection*{Fermented vegetable foods}
Many vegetables such as leafy greens, radish, cucumbers, and young edible bamboo shoots, are perishable and sesasonal, therefore they are traditionally fermented into edible products to ensure longer shelflife and safe consumption \cite*{L1-DiversityMicro}. The main microorganisms involved in the fermentation of vegetables are LAB, which are responsible for the acidification of the product. The most common LAB in fermented vegetables are \textit{Lactobacillus} and \textit{Pediococcus}, followed by \textit{Leuconostoc}, \textit{Weissella}, \textit{Tetragenococcus}, and \textit{Lactococcus} \cite*{L1-DiversityMicro}.
Some examples of fermented vegetable products are listed below in table \ref*{tab:Fermented_vegetables} \cite*{L1-DiversityMicro}:

\begin{table}[h]
    \centering
    \caption{Examples of fermented vegetable products}
    \label{tab:Fermented_vegetables}
    \begin{tabular}{c|>{\centering\arraybackslash}m{0.5\textwidth}}
        \textbf{Food} & \textbf{Description} \\
        \hline
        Kimchi & A Korean fermented vegetable product, with a characterized microbial profile of LAB \\
        Sauerkraut & A German fermented cabbage product, with a reported species of LAB \\
        Gundurk & have a native population of LAB. Made by fermenting leafy greans which then sun dried to create a sour and slightly bitter flavor \\
    \end{tabular}
\end{table}

\subsubsection*{Fermented soybeans and other legumes}
There are two types of fermented soybean foods, bacillus-fermented and mould-fermented. The bacillus-fermented foods are fermented by \textit{Bacillus} spp. (mostly \textit{B. subtilis}), and are characterized by stickiness. The mould-fermented foods are fermented by filamentous molds, mostly \textit{Aspergillus}, \textit{Mucor}, and \textit{Rhizopus}.

These types of foods are concentrated in the "Kinema-Natto-Thua Nao (KNT)-triangle" region, which includes Japan, east Nepal, north-east India, and northern Thailand. Examples of bacillus-, mould-, and other fermented foods are listed in table \ref*{tab:fermented_soybean} \cite*{L1-DiversityMicro}.

\begin{table}[H]
    \centering
    \caption{Examples of Bacillus-, mould-, and other fermented foods}
    \label{tab:fermented_soybean}
    \begin{tabularx}{\textwidth}{p{0.33\textwidth}|p{0.33\textwidth}|p{0.33\textwidth}}

        \multicolumn{1}{c|}{\textbf{Bacillus-fermented}} & \multicolumn{1}{c|}{\textbf{Mould-fermented}} & \multicolumn{1}{c}{\textbf{Fermented foods}} \\
        \hline
        \begin{tabular}[t]{p{0.16\textwidth}|p{0.16\textwidth}}
            \textbf{Foodstuff} & \textbf{Origin} \\
            Natto & Japan \\
            Kinema & India \\
            Chungkokjang & Korea \\
            Thua nao & Thailand \\
            Pekok & Myanmar \\
            Sieng & Cambodia \\
        \end{tabular} &
        \begin{tabular}[t]{p{0.16\textwidth}|p{0.16\textwidth}}
            \textbf{Foodstuff} & \textbf{Origin} \\
            Miso and shoyu & Japan \\
            Tempe & Indonesia \\
            Douchi and sufu & China \\
            Doenjang & Korea \\
            Pekok & Myanmar \\
            Sieng & Cambodia \\
        \end{tabular} &
        \begin{tabular}[t]{p{0.16\textwidth}|p{0.16\textwidth}}
            \textbf{Foodstuff} & \textbf{Origin} \\
            Locust beans & Africa \\
            Black-grams & India \\
            Maseura & Nepal \\
            \multicolumn{2}{c}{} \\
        \end{tabular} \\
    \end{tabularx}
\end{table}

\subsubsection*{Fermented meat products}
Overall, fermented meat products are divided into two categories, whole meat pieces (e.g. jamón serrano) and chopped meat pieces (e.g. chorizo). The main microbial groups invovled in fermenting meats are LAB, coagulase-negative cocci, and yeasts. The LAB are responsible for the acidification of the meat, while the coagulase-negative cocci are responsible for the ripening of the meat. The yeasts are responsible for the development of the flavor and aroma of the meat \cite*{L1-DiversityMicro}.

\subsubsection*{Fermented fish products}
Traditionally fish is preserved by either fermentation, sun/smoke drying
and salting. Both bacteria and yeasts have been reported from these preservation methods. 

\subsubsection*{Fermented miscellaneous products}
The article discusses various fermented foods and the microorganisms involved in their production. These foods include \cite*{L1-DiversityMicro}:

\begin{highlight}
    \begin{itemize}
        \item Vinegar
        \subitem Produced from sugar or ethanol-containing substrates and hydrolyzed starchy materials by 
        \subitem aerobic conversion to acetic acid.
        \subitem Dominant microorganisms include \textit{Acetobacter} species and yeast species such as \textit{Candida} and 
        \subitem \textit{Saccharomyces}.
        \vspace*{0.5em}

        \item Fermented Teas
        \subitem Such as miang, puer tea, fuzhuan brick, and kombucha.
        \subitem Microorganisms involved include \textit{Aspergillus}, \textit{Penicillium}, and \textit{Saccharomyces} species.
        \vspace*{0.5em}

        \item Nata
        \subitem A delicacy from the Philippines, produced by \textit{Acetobacter xylinum}.
        \subitem Two types of nata are known: nata de piña and nata de coco.
        \vspace*{0.5em}

        \item Chocolate
        \subitem Produced from cocoa bean fermentation. 
        \subitem The predominant microorganisms are \textit{Acetobacter pasteurianus}, \textit{Lactobacillus plantarum}, and 
        \subitem \textit{Saccharomyces cerevisiae}.

    \end{itemize}
\end{highlight}
Continues on the next page.

\begin{highlight}
    \begin{itemize}
        \item Pidan
        \subitem A preserved egg prepared from alkali-treated fresh duck eggs, with a strong hydrogen 
        \subitem sulfide and ammonia smell.
        \subitem Dominant microorganisms include \textit{Bacillus cereus}, \textit{Staphylococcus cohnii}, and \textit{Staphylococcus}
        \subitem \textit{epidermidis}.
    \end{itemize}    
\end{highlight}

Some common microorganisms found in miscellaneous fermented foods include \cite*{L1-DiversityMicro}:

\begin{highlight}
    \begin{itemize}
        \item LAB; such as \textit{Lb. fermentum}, \textit{Lb. plantarum}, and \textit{Weissella} species.
        \item Acetobacter; such as \textit{Acetobacter pasteurianus} and \textit{Acetobacter xylinum}.
        \item Yeast; such as \textit{Saccharomyces}, \textit{Candida}, and \textit{Hanseniaspora} species.
        \item Mold; such as \textit{Aspergillus} and \textit{Penicillium} species.
    \end{itemize}
\end{highlight}

\subsubsection*{Amolytic starters}
Amylolytic starters are traditional cultures of microorganisms used to produce alcoholic beverages and fermented foods from starchy materials. These starters are composed of consortia of filamentous molds, amylolytic, and alcohol-producing yeasts and LAB, and are typically made with rice or wheat as the base \cite*{L1-DiversityMicro}.


The article described some of the different types of amylolytic starters which are used in various countries, including \cite*{L1-DiversityMicro}:

\begin{highlight}
    \begin{itemize}
        \item \textbf{Marcha} (India and Nepal)
        \item \textbf{Hamei, humao, phab} (India)
        \item \textbf{Men} (Vietnam)
        \item \textbf{Ragi} (Indonesia)
        \item \textbf{Bubod} (Philippines)
        \item \textbf{Chiu/chu} (China and Taiwan)
        \item \textbf{Loogpang} (Thailand)
        \item \textbf{Mae/dombae/buh/puh} (Cambodia)
        \item \textbf{Nuruk} (Korea)
    \end{itemize}
\end{highlight}


Furthermore the article named some of the common microorganisms involved in amylolytic starters which included \cite*{L1-DiversityMicro}:

\begin{highlight}
    \begin{itemize}
        \item Filamentous molds; such as \textit{Mucor circinelloides}, \textit{Rhizopus chinensis}, and \textit{Aspergillus oryzae}.
        \item Yeasts; such as \textit{Saccharomyces cerevisiae}, \textit{Pichia anomala}, and \textit{Candida glabrata}.
        \item LAB; such as \textit{Pediococcus pentosaceus}, \textit{Lactobacillus bifermentans}, and \textit{Lactobacillus brevis}.
        \item Amylase-producing bacilli; such as \textit{Bacillus subtilis} and \textit{Bacillus amyloliquefaciens}.
        \item Acetic acid bacteria; such as \textit{Acetobacter orientalis} and \textit{Acetobacter pasteurianus}.
    \end{itemize}
\end{highlight}

\subsubsection*{Alcoholic beverages}
Alcoholic beverages can be classified into 10 types based on their production methods and ingredients. These types include \cite*{L1-DiversityMicro}:

\begin{itemize}
    \item \textbf{Non-distilled and unfiltered beverages produced by amylolytic starters}:, such as kodo ko jaanr (fermented finger millets) and bhaati jaanr (fermented rice) of India and Nepal.
    \item \textbf{Non-distilled and filtered beverages produced by amylolytic starters}:, such as saké of Japan.
    \item \textbf{Distilled beverages produced by amylolytic starters}:, such as shochu of Japan and soju of Korea.
    \item \textbf{Beverages produced by human saliva}: Such as chicha of Peru, which involves the use of amylase in human saliva.
    \item \textbf{Mono-fermentation beverages}: Produced by a single-strain fermentation, such as beer.
    \item \textbf{Honey-based beverages}: Such as tej of Ethiopia.
    \item \textbf{Plant-based beverages}: Such as pulque of Mexico, toddy of India, and kanji of India.
    \item \textbf{Malted beverages}: Produced by malting (germination), such as sorghum (“Bantu”) beer of South Africa, pito of Nigeria and Ghana, and tchoukoutou of Benin.
    \item \textbf{Fruit-based beverages without distillation}: Such as wine and cider.
    \item \textbf{Distilled fruit and cereal beverages}: Such as whisky and brandy.
\end{itemize}


\subsection{Article 3 - Molecular biology of the cell; chapter 6}

\subsubsection*{In this chapter}
In this chapter of the book, the following topics will be further described:

\textbf{The central dogma of molecular biology};
The discovery of the structure of DNA in the 1950s has led to significant progress in cell and molecular biology. We now know the complete genome sequences for thousands of organisms, including humans and our extinct relatives, such as the Neanderthals. The flow of genetic information in cells is from DNA to RNA to protein, a principle known as the central dogma of molecular biology \cite*{L1-Chapter6}.

\textbf{Variations in the central dogma};
While the central dogma is universal, there are important variations between organisms in the way information flows from DNA to protein. In eukaryotic cells, RNA transcripts are subject to processing steps in the nucleus, including RNA splicing, before they are translated into protein \cite*{L1-Chapter6}.

\textbf{The role of RNA};
RNA plays a crucial role in the cell, not only as a template for protein synthesis but also as a structural and catalytic molecule. Some RNAs fold into precise three-dimensional structures, while others act as regulators of gene expression \cite*{L1-Chapter6}.

\textbf{The complexity of genomes};
The genomes of most multicellular organisms are surprisingly disorderly, reflecting their chaotic evolutionary histories. Genes are often arranged in a long string of alternating short exons and long introns, with small bits of DNA sequence that code for protein interspersed with large blocks of seemingly meaningless DNA \cite*{L1-Chapter6}.

\textbf{Decoding genomes};
Decoding genomes is a complex task, even with the aid of powerful computers. Cells in our body can automatically locate the beginning and end of genes and decipher when and where each gene is expressed, but researchers still face significant challenges in understanding how genomes are decoded and used \cite*{L1-Chapter6}.

\textbf{The challenges of cell biology};
Despite significant progress in cell and molecular biology, there is still much to be discovered about how the information stored in an organism's genome produces even the simplest unicellular bacterium, let alone how it directs the development of a human with approximately 30,000 genes. The next generation of cell biologists will face many fascinating challenges in understanding the complexities of genomes and how they are decoded and used \cite*{L1-Chapter6}.

\subsubsection*{From DNA to RNA}
\textbf{Transcription and translation} are the means by which cells read out, or express, the genetic instructions in their genes \cite*{L1-Chapter6}.

\textbf{Regulation of gene expression} is when the cells can synthesize a large amount of protein from a gene when necessary, and genes can be transcribed and translated with different efficiencies, allowing the cell to make vast quantities of some proteins and tiny amounts of others \cite*{L1-Chapter6}.

\subsubsection*{RNA molecules are single-stranded}
\textbf{Transcription} is the first step a cell takes in reading out a needed part of its genetic instructions is to copy a particular portion of its DNA nucleotide sequence—a gene—into an RNA nucleotide sequence \cite*{L1-Chapter6}.

\textbf{RNA structure} is a linear polymer made of four different types of nucleotide subunits linked together by phosphodiester bonds. It differs from DNA chemically in two respects: it contains the sugar ribose and the base uracil (U) instead of thymine (T) \cite*{L1-Chapter6}.

\textbf{RNA folding} is single-stranded and can fold up into a particular shape, allowing some RNA molecules to have precise structural and catalytic functions \cite*{L1-Chapter6}.

\subsubsection*{Transcription produces RNA complementary to one strand of DNA}
\textbf{Transcription process} begins with the opening and unwinding of a small portion of the DNA double helix to expose the bases on each DNA strand. One of the two strands of the DNA double helix then acts as a template for the synthesis of an RNA molecule \cite*{L1-Chapter6}.

\textbf{RNA synthesis} happens when nucleotide sequence of the RNA chain is determined by the complementary base-pairing between incoming nucleotides and the DNA template. The RNA chain is elongated one nucleotide at a time, and it has a nucleotide sequence that is exactly complementary to the strand of DNA used as the template \cite*{L1-Chapter6}.

\textbf{Differs from DNA replication} in several crucial ways. Unlike a newly formed DNA strand, the RNA strand does not remain hydrogen-bonded to the DNA template strand. Instead, the RNA chain is displaced and the DNA helix re-forms, releasing the RNA molecules as single strands \cite*{L1-Chapter6}.

\subsubsection*{RNA polymerase carry out transcription}
\textbf{RNA polymerases} are enzymes that perform transcription. They catalyze the formation of the phosphodiester bonds that link the nucleotides together to form a linear chain \cite*{L1-Chapter6}.

\textbf{Mechanism of RNA polymerase} involves the RNA polymerase moving stepwise along the DNA, unwinding the DNA helix just ahead of the active site for polymerization to expose a new region of the template strand for complementary base-pairing, and extending the growing RNA chain by one nucleotide at a time in the 5'-to-3' direction \cite*{L1-Chapter6}.

\textbf{Characteristics of RNA polymerase} include the ability to start an RNA chain without a primer, being absolutely processive, and having a modest proofreading mechanism to correct errors in the growing RNA chain \cite*{L1-Chapter6}.

\textbf{Comparison with DNA polymerase} reveals that RNA polymerase is less accurate, with an error rate of about one mistake for every 10\textsuperscript{4} nucleotides copied into RNA, despite catalyzing essentially the same chemical reaction as DNA polymerase \cite*{L1-Chapter6}.

\subsubsection*{RNA polymerase binds to DNA at specific sites}
\textbf{Types of RNA} include messenger RNA (mRNA), which directs the synthesis of proteins, and noncoding RNAs, which do not code for protein and serve as enzymatic, structural, and regulatory components \cite*{L1-Chapter6}.

\textbf{Functions of noncoding RNAs} include directing the splicing of pre-mRNA to form mRNA, forming the core of ribosomes, and serving as adaptors that select amino acids and hold them in place on a ribosome for incorporation into protein\cite*{L1-Chapter6}.

\textbf{Transcription units} are segments of DNA that are transcribed into RNA. In eukaryotes, a transcription unit typically carries the information of just one gene, while in bacteria, a set of adjacent genes is often transcribed as a unit \cite*{L1-Chapter6}.

\textbf{RNA composition in cells} reveals that RNA makes up a few percent of a cell's dry weight, with most of the RNA being rRNA, and mRNA comprising only 3–5\% of the total RNA in a typical mammalian cell \cite*{L1-Chapter6}.

\subsubsection*{Signals encoded in DNA tell RNA poluymerase where to start and stop}
\textbf{Initiation of transcription} is a crucial step in gene expression, where RNA polymerase recognizes where to start and stop transcribing a gene. The bacterial RNA polymerase core enzyme is a multisubunit complex that synthesizes RNA using the DNA template as a guide, with the help of the sigma ($\sigma$) factor \cite*{L1-Chapter6}.

During \textbf{transcription bubble formation}, the RNA polymerase holoenzyme opens up the double helix to expose a short stretch of nucleotides on each strand. The first ten or so nucleotides of RNA are synthesized using a "scrunching" mechanism, and the polymerase then moves down the DNA, synthesizing RNA in a stepwise fashion \cite*{L1-Chapter6}.

Chain elongation continues until the enzyme encounters the \textbf{\textit{Termination}} signal, which consists of a string of A-T nucleotide pairs preceded by a twofold symmetric DNA sequence. The formation of a hairpin structure through Watson–Crick base-pairing helps to disengage the RNA transcript from the active site, marking the end of transcription \cite*{L1-Chapter6}.

\subsubsection*{Transcription Start and Stop Signals Are Heterogeneous in Nucleotide Sequence}
Transcription initiation and termination involve a series of structural transitions in protein, DNA, and RNA molecules, and the signals encoded in DNA that specify these transitions are often difficult to recognize. Despite this, researchers have identified common features among bacterial promoters, which are often summarized in the form of a \textbf{consensus sequence}. A consensus sequence is derived by comparing many sequences with the same basic function and tallying up the most common nucleotides found at each position \cite*{L1-Chapter6}.

\textbf{Promoter Strength and Sequence Variation}
The DNA sequences of individual bacterial promoters differ in ways that determine their strength, with promoters for genes that code for abundant proteins being much stronger than those associated with genes that encode rare proteins. The nucleotide sequences of their promoters are responsible for these differences \cite*{L1-Chapter6}.

\textbf{Terminator Sequences}
Like bacterial promoters, transcription terminators also have a wide range of sequences, with the potential to form a simple hairpin RNA structure being the most important common feature. Terminator sequences are even more heterogeneous than promoter sequences \cite*{L1-Chapter6}.

\textbf{Challenges in Locating Promoters and Terminators}
Despite our knowledge of bacterial promoters and terminators, their variation in nucleotide sequence makes it difficult to definitively locate them simply by analysis of the nucleotide sequence of a genome. Additional information, some of it from direct experimentation, is often needed to locate and accurately interpret the short DNA signals in genomes \cite*{L1-Chapter6}.

\textbf{Promoter Orientation and Template Strand}
Promoter sequences are asymmetric, ensuring that RNA polymerase can bind in only one orientation. The promoter orientation specifies the strand to be used as a template, and genome sequences reveal that the DNA strand used as the template for RNA synthesis varies from gene to gene, depending on the orientation of the promoter \cite*{L1-Chapter6}.

\textbf{Transcription Initiation in Eukaryotes Requires Many Proteins}
Eukaryotic nuclei have three types of RNA polymerase, with RNA polymerase II transcribing most genes, including those that encode proteins \cite*{L1-Chapter6}.

\subsubsection*{Key Differences from Bacterial RNA Polymerase}
Eukaryotic RNA polymerase II differs from bacterial RNA polymerase in two main ways \cite*{L1-Chapter6}:
\begin{highlight}
    \begin{itemize}
        \item It requires many transcription-initiation factors, collectively called the general transcription factors.
        \item It must initiate transcription on DNA packaged into \textit{nucleosomes} and higher-order chromatin structures.
    \end{itemize}
\end{highlight}

\subsubsection*{RNA Polymerase II Requires a Set of General Transcription Factors}
Eukaryotic RNA polymerase II requires a set of general transcription factors, denoted as TFIIA, TFIIB, TFIIC, TFIID, and so on, to position correctly at the promoter and initiate transcription \cite*{L1-Chapter6}.

\textbf{Assembly of General Transcription Factors}
The assembly process begins when TFIID binds to the TATA sequence, a short double-helical DNA sequence primarily composed of T and A nucleotides. The binding of TFIID causes a large distortion in the DNA, which serves as a physical landmark for the location of an active promoter \cite*{L1-Chapter6}.

\textbf{Role of TFIIH}
TFIIH, the most complicated of the general transcription factors, performs several enzymatic steps needed for the initiation of transcription, including unwinding the DNA and exposing the template strand \cite*{L1-Chapter6}.

\textbf{Phosphorylation of RNA Polymerase II}
The phosphorylation of the C-terminal domain (CTD) of RNA polymerase II, specifically the serine located at the fifth position in the repeat sequence (Ser5), is a key step in the transition from initiation to elongation. This phosphorylation allows the polymerase to disengage from the cluster of general transcription factors and undergo conformational changes that tighten its interaction with DNA \cite*{L1-Chapter6}.

\textbf{Release of General Transcription Factors} 
Once the polymerase II has begun elongating the RNA transcript, most of the general transcription factors are released from the DNA, making them available to initiate another round of transcription with a new RNA polymerase molecule \cite*{L1-Chapter6}.

\subsubsection*{Polymerase II Also Requires Activator, Mediator, and Chromatin-Modifying Proteins}
In addition to general transcription factors, RNA polymerase II requires several other proteins to initiate transcription in a eukaryotic cell, including \cite*{L1-Chapter6}:
\begin{highlight}
    \begin{itemize}
        \item \textbf{Transcriptional activators}, which bind to specific sequences in DNA (called enhancers) and help to attract RNA polymerase II to the start point of transcription.
        \item \textbf{Mediator}, a large protein complex that allows activator proteins to communicate properly with the polymerase II and with the general transcription factors.
    \end{itemize}
\end{highlight}
\textit{Continues on the next page.}
\begin{highlight}
    \begin{itemize}
        \item \textbf{Chromatin-modifying enzymes}, including chromatin remodeling complexes and histone-modifying enzymes, which increase access to the DNA in chromatin and facilitate the assembly of the transcription initiation machinery onto DNA.
    \end{itemize}
\end{highlight}

\textbf{Assembly of Transcription Initiation Machinery}
The assembly of the transcription initiation machinery onto DNA involves the recruitment of many proteins (well over 100 individual subunits) and does not follow a prescribed pathway. The order of assembly differs from gene to gene, and some protein complexes may be brought to DNA as preformed subassemblies \cite*{L1-Chapter6}.

\textbf{Release of RNA Polymerase II}
To begin transcribing, RNA polymerase II must be released from the large complex of proteins. This release often requires the in situ proteolysis of the activator protein \cite*{L1-Chapter6}.

\subsubsection*{Transcription Elongation in Eukaryotes Requires Accessory Proteins}
Once RNA polymerase has initiated transcription, it requires accessory proteins to facilitate elongation. These proteins include \cite*{L1-Chapter6}:

\begin{highlight}
    \begin{itemize}
        \item \textbf{Elongation factors}, which decrease the likelihood that RNA polymerase will dissociate before it reaches the end of a gene.
        \item \textbf{ATP-dependent chromatin remodeling complexes}, which help RNA polymerase move through chromatin structure by either moving with the polymerase or rescuing stalled polymerases.
        \item \textbf{Histone chaperones}, which partially disassemble nucleosomes in front of a moving RNA polymerase and assemble them behind.
    \end{itemize}
\end{highlight}

\textbf{Modification of Histones During Transcription Elongation}
As RNA polymerase moves along a gene, enzymes bound to it modify the histones, leaving behind a record of where the polymerase has been. This information may aid in transcribing a gene over and over again once it has become active for the first time, and may also be used to coordinate transcription elongation with RNA processing \cite*{L1-Chapter6}.

\subsubsection*{Transcription Creates Superhelical Tension}
\textbf{DNA Supercoiling} is a conformation that DNA adopts in response to superhelical tension, which can be created by opening or unwinding a double-helical DNA molecule \cite*{L1-Chapter6}.

\textbf{Effect of RNA Polymerase on DNA Supercoiling} is to create superhelical tension as it moves along a stretch of DNA that is anchored at its ends, generating positive superhelical tension in the DNA in front of it and negative helical tension behind it \cite*{L1-Chapter6}.

\textbf{Consequences of DNA Supercoiling} include making the DNA helix more difficult to open, but also facilitating the partial unwrapping of the DNA in nucleosomes in eukaryotes \cite*{L1-Chapter6}.

\textbf{Removal of Superhelical Tension} is achieved by DNA topoisomerase enzymes in eukaryotes, while in bacteria, DNA gyrase uses the energy of ATP hydrolysis to pump supercoils continuously into the DNA, maintaining the DNA under constant tension \cite*{L1-Chapter6}.

\subsubsection*{Transcription Elongation in Eukaryotes Is Tighly Coupled to RNA Processing}
\textbf{Eukaryotic mRNA Synthesis} involves several steps beyond transcription, including covalent modification of the ends of the RNA and removal of intron sequences by RNA splicing \cite*{L1-Chapter6}.

\textbf{Modification of Eukaryotic mRNAs} includes capping on the 5' end and polyadenylation of the 3' end, allowing the cell to assess the integrity of the mRNA molecule before exporting it from the nucleus and translating it into protein \cite*{L1-Chapter6}.

\textbf{RNA Splicing} joins together the different portions of a protein-coding sequence, enabling eukaryotes to synthesize several different proteins from the same gene \cite*{L1-Chapter6}. 

\textbf{Coupling of RNA Processing to Transcription Elongation} is achieved through the phosphorylation of the RNA polymerase II tail, which allows a new set of proteins to associate with the RNA polymerase and function in transcription elongation and RNA processing \cite*{L1-Chapter6}.

\subsubsection*{RNA Capping Is the First Modification of Eukaryotic Pre-mRNAs}
\textbf{5' Capping of Eukaryotic mRNAs} involves the addition of a modified guanine nucleotide to the 5' end of the new RNA molecule, which is performed by three enzymes acting in succession \cite*{L1-Chapter6}.

\textbf{Mechanism of 5' Capping} includes the removal of a phosphate from the 5' end of the nascent RNA, the addition of a GMP in a reverse linkage, and the addition of a methyl group to the guanosine \cite*{L1-Chapter6}.

\textbf{Function of the 5'-Methyl Cap} is to signify the 5'' end of eukaryotic mRNAs, helping the cell to distinguish mRNAs from other types of RNA molecules, and to facilitate further processing and export of the mRNA \cite*{L1-Chapter6}.

\textbf{Binding of the 5'-Methyl Cap} to a protein complex called CBC (cap-binding complex) in the nucleus, which helps the mRNA to be further processed and exported \cite*{L1-Chapter6}.

\subsubsection*{RNA Splicing Removes Intron Sequences from Newly Transcribed Pre-mRNAs}
\textbf{Eukaryotic Gene Structure} is characterized by the presence of noncoding intervening sequences (introns) that interrupt the protein-coding sequences (exons) \cite*{L1-Chapter6}.

\textbf{RNA Splicing} is the process by which intron sequences are removed from the newly synthesized RNA, and the exons are joined together through two sequential phosphoryl-transfer reactions known as transesterifications \cite*{L1-Chapter6}.

\textbf{Mechanism of RNA Splicing} involves a complex machinery consisting of five additional RNA molecules and several hundred proteins, which hydrolyzes many ATP molecules per splicing event to ensure accurate and flexible splicing \cite*{L1-Chapter6}.

\textbf{Advantages of RNA Splicing} include facilitating the emergence of new and useful proteins over evolutionary time scales by allowing genetic recombination to combine the exons of different genes, and enabling eukaryotes to increase the coding potential of their genomes by producing a corresponding set of different proteins from the same gene through alternative splicing \cite*{L1-Chapter6}.

\subsubsection*{Nucleotide Sequences Signal Where Splicing Occurs}
\textbf{Recognition of Splicing Sites} requires the splicing machinery to identify three portions of the precursor RNA molecule: the 5' splice site, the 3' splice site, and the branch point in the intron sequence \cite*{L1-Chapter6}.

\textbf{Consensus Sequences for Splicing} are short nucleotide sequences that are similar from intron to intron and provide cues for where splicing is to take place, but can accommodate extensive sequence variability \cite*{L1-Chapter6}.

\textbf{Challenges in Deciphering Genome Sequences} arise from the high variability of splicing consensus sequences and the possibility of alternative splicing, making it difficult to predict protein sequences solely from a genome sequence and identify all of the genes in a complete genome sequence \cite*{L1-Chapter6}.

\subsubsection*{Nucleotide Sequences Signal Where Splicing Occurs}
\textbf{Recognition of Splicing Sites} requires the splicing machinery to identify three portions of the precursor RNA molecule: the 5' splice site, the 3' splice site, and the branch point in the intron sequence \cite*{L1-Chapter6}.

\textbf{Consensus Sequences for Splicing} are short nucleotide sequences that are similar from intron to intron and provide cues for where splicing is to take place, but can accommodate extensive sequence variability \cite*{L1-Chapter6}.

\textbf{Challenges in Deciphering Genome Sequences} arise from the high variability of splicing consensus sequences and the possibility of alternative splicing, making it difficult to predict protein sequences solely from a genome sequence and identify all of the genes in a complete genome sequence \cite*{L1-Chapter6}.

\subsubsection*{RNA Splicing Is Performed by the Spliceosome}
\textbf{Role of RNA Molecules in RNA Splicing} involves specialized RNA molecules, known as snRNAs (small nuclear RNAs), that recognize the nucleotide sequences that specify where splicing is to occur and catalyze the chemistry of splicing \cite*{L1-Chapter6}.

\textbf{Composition of the Spliceosome} includes five snRNAs (U1, U2, U4, U5, and U6) complexed with at least seven protein subunits to form an snRNP (small nuclear ribonucleoprotein), which forms the core of the spliceosome \cite*{L1-Chapter6}.

\textbf{Mechanism of Spliceosome Assembly} involves the recognition of the 5' splice junction, the branch-point site, and the 3' splice junction through base-pairing between the snRNAs and the consensus RNA sequences in the pre-mRNA substrate, and the dynamic assembly and rearrangement of the spliceosome components during the splicing reaction \cite*{L1-Chapter6}.

\textbf{Spliceosome Structure and Function} is complex and dynamic, with some scientists believing that the spliceosome is a preexisting, loose assembly of all the components that captures, splices, and releases RNA as a coordinated unit, undergoing extensive rearrangements each time a splice is made \cite*{L1-Chapter6}.

\subsubsection*{The Spliceosome Uses ATP Hydrolysis to Produce a Complex Series of RNA-RNA Rearrangements}
\textbf{Energy Requirements for RNA Splicing} reveals that ATP hydrolysis is not required for the chemistry of RNA splicing itself, but is necessary for the assembly and rearrangements of the spliceosome \cite*{L1-Chapter6}.

\textbf{Role of ATP Hydrolysis in Spliceosome Assembly} involves the use of energy from ATP hydrolysis to break existing RNA-RNA interactions and allow the formation of new ones, enabling the splicing signals on the pre-RNA to be examined by snRNPs several times during the course of splicing \cite*{L1-Chapter6}.

\textbf{Purpose of Spliceosome Rearrangements} is to increase the overall accuracy of splicing by allowing the spliceosomes to check and recheck the splicing signals, and to create the active sites for the two transesterification reactions \cite*{L1-Chapter6}.

\textbf{Catalytic Sites of the Spliceosome} are formed by both protein and RNA molecules, with the RNA molecules catalyzing the actual chemistry of splicing \cite*{L1-Chapter6}.

\textbf{Disassembly of the Spliceosome} requires another series of RNA-RNA rearrangements that require ATP hydrolysis, returning the snRNAs to their original configuration so that they can be used again in a new reaction \cite*{L1-Chapter6}.

\textbf{Exon Junction Complex (EJC)} is a set of proteins that bind to the mRNA near the position formerly occupied by the intron, marking the site of a successful splicing event and influencing the subsequent fate of the mRNA \cite*{L1-Chapter6}.

\subsubsection*{ds RNA Can Be Processed by Dicer to Produce siRNAs}
\textbf{Energy Requirements for RNA Splicing} reveals that ATP hydrolysis is not required for the chemistry of RNA splicing itself, but is necessary for the assembly and rearrangements of the spliceosome \cite*{L1-Chapter6}.

\textbf{Role of ATP Hydrolysis in Spliceosome Assembly} involves the use of energy from ATP hydrolysis to break existing RNA-RNA interactions and allow the formation of new ones, enabling the splicing signals on the pre-RNA to be examined by snRNPs several times during the course of splicing \cite*{L1-Chapter6}.

\textbf{Purpose of Spliceosome Rearrangements} is to increase the overall accuracy of splicing by allowing the spliceosomes to check and recheck the splicing signals, and to create the active sites for the two transesterification reactions \cite*{L1-Chapter6}.

\textbf{Catalytic Sites of the Spliceosome} are formed by both protein and RNA molecules, with the RNA molecules catalyzing the actual chemistry of splicing \cite*{L1-Chapter6}.

\textbf{Disassembly of the Spliceosome} requires another series of RNA-RNA rearrangements that require ATP hydrolysis, returning the snRNAs to their original configuration so that they can be used again in a new reaction \cite*{L1-Chapter6}.

\textbf{Exon Junction Complex (EJC)} is a set of proteins that bind to the mRNA near the position formerly occupied by the intron, marking the site of a successful splicing event and influencing the subsequent fate of the mRNA \cite*{L1-Chapter6}.
